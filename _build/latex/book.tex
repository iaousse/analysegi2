%% Generated by Sphinx.
\def\sphinxdocclass{report}
\documentclass[letterpaper,10pt,french]{sphinxmanual}
\ifdefined\pdfpxdimen
   \let\sphinxpxdimen\pdfpxdimen\else\newdimen\sphinxpxdimen
\fi \sphinxpxdimen=.75bp\relax

\PassOptionsToPackage{warn}{textcomp}
\usepackage[utf8]{inputenc}
\ifdefined\DeclareUnicodeCharacter
% support both utf8 and utf8x syntaxes
  \ifdefined\DeclareUnicodeCharacterAsOptional
    \def\sphinxDUC#1{\DeclareUnicodeCharacter{"#1}}
  \else
    \let\sphinxDUC\DeclareUnicodeCharacter
  \fi
  \sphinxDUC{00A0}{\nobreakspace}
  \sphinxDUC{2500}{\sphinxunichar{2500}}
  \sphinxDUC{2502}{\sphinxunichar{2502}}
  \sphinxDUC{2514}{\sphinxunichar{2514}}
  \sphinxDUC{251C}{\sphinxunichar{251C}}
  \sphinxDUC{2572}{\textbackslash}
\fi
\usepackage{cmap}
\usepackage[T1]{fontenc}
\usepackage{amsmath,amssymb,amstext}
\usepackage{babel}



\usepackage{times}
\expandafter\ifx\csname T@LGR\endcsname\relax
\else
% LGR was declared as font encoding
  \substitutefont{LGR}{\rmdefault}{cmr}
  \substitutefont{LGR}{\sfdefault}{cmss}
  \substitutefont{LGR}{\ttdefault}{cmtt}
\fi
\expandafter\ifx\csname T@X2\endcsname\relax
  \expandafter\ifx\csname T@T2A\endcsname\relax
  \else
  % T2A was declared as font encoding
    \substitutefont{T2A}{\rmdefault}{cmr}
    \substitutefont{T2A}{\sfdefault}{cmss}
    \substitutefont{T2A}{\ttdefault}{cmtt}
  \fi
\else
% X2 was declared as font encoding
  \substitutefont{X2}{\rmdefault}{cmr}
  \substitutefont{X2}{\sfdefault}{cmss}
  \substitutefont{X2}{\ttdefault}{cmtt}
\fi


\usepackage[Sonny]{fncychap}
\ChNameVar{\Large\normalfont\sffamily}
\ChTitleVar{\Large\normalfont\sffamily}
\usepackage[,numfigreset=1,mathnumfig]{sphinx}

\fvset{fontsize=\small}
\usepackage{geometry}


% Include hyperref last.
\usepackage{hyperref}
% Fix anchor placement for figures with captions.
\usepackage{hypcap}% it must be loaded after hyperref.
% Set up styles of URL: it should be placed after hyperref.
\urlstyle{same}


\usepackage{sphinxmessages}




\title{Cours Analyse II}
\date{mars 20, 2022}
\release{}
\author{Iaousse M\textquotesingle{}barek}
\newcommand{\sphinxlogo}{\vbox{}}
\renewcommand{\releasename}{}
\makeindex
\begin{document}

\ifdefined\shorthandoff
  \ifnum\catcode`\=\string=\active\shorthandoff{=}\fi
  \ifnum\catcode`\"=\active\shorthandoff{"}\fi
\fi

\pagestyle{empty}
\sphinxmaketitle
\pagestyle{plain}
\sphinxtableofcontents
\pagestyle{normal}
\phantomsection\label{\detokenize{intro::doc}}


\sphinxAtStartPar
Le présent cours contiendra :
\begin{itemize}
\item {} 
\sphinxAtStartPar
Intégration

\item {} 
\sphinxAtStartPar
Intégration et dérivation

\item {} 
\sphinxAtStartPar
Intégration sur un intervale quelconque

\end{itemize}


\chapter{Intégration}
\label{\detokenize{integration:integration}}\label{\detokenize{integration::doc}}
\sphinxAtStartPar
Le présent Chapitre contiendra :
\begin{itemize}
\item {} 
\sphinxAtStartPar
Integrale des fonctions escalier

\item {} 
\sphinxAtStartPar
Integrale des fonctions continues par morceaux

\item {} 
\sphinxAtStartPar
Propriete de l’integrale

\item {} 
\sphinxAtStartPar
Somme de Reiman

\item {} 
\sphinxAtStartPar
Fonctions continue par morceaux sur un intervalle

\end{itemize}


\section{Intégrale des fonctions en escalier}
\label{\detokenize{fe:integrale-des-fonctions-en-escalier}}\label{\detokenize{fe::doc}}
\sphinxAtStartPar
\(a\) et \(b\) désignent deux réels tels que \(a < b \). Toutes    les fonctions sont supposées définies sur \([a, b]\) et a valeurs réelles.
Le but de l’intégration est de définir un nombre qui, pour une fonction \(f\) positive sur un segment \([a, b]\) , mesure l’aire délimitée par sa courbe représentative, l’axe des abscisses et les deux droites d’équations \(x = a\) et \(x = b\).
Ce nombre sera appelé intégrale de \(f\) sur \([a, b]\) et notée :
\begin{equation*}
\begin{split}
\int_a^b f(x)dx
\end{split}
\end{equation*}

\subsection{Subdivision d’un segment}
\label{\detokenize{fe:subdivision-d-un-segment}}
\begin{sphinxadmonition}{note}{Définition 1}
\begin{itemize}
\item {} 
\sphinxAtStartPar
On appelle subdivision de \([a, b]\) toute famille \(u=(x_i)_{i=0}^n\) telle que \(n \in \mathbb{N}^\star\) et

\end{itemize}
\begin{equation*}
\begin{split}
a=x_0 < x_1 < \ldots < x_n=b
\end{split}
\end{equation*}\begin{itemize}
\item {} 
\sphinxAtStartPar
On appelle pas ou module de la subdivision \(u=(x_i)_{i=0}^n\), le reel

\end{itemize}
\begin{equation*}
\begin{split}
\delta(u)=\max_{i \in [1, n]}(x_i - x_{i-1})
\end{split}
\end{equation*}\end{sphinxadmonition}

\begin{sphinxadmonition}{note}{Exemple}

\sphinxAtStartPar
Une subdivision \((x_i)_{i=0}^n\) ou \(n\) un entier naturel non nul est dite a pas constant si:
\begin{equation*}
\begin{split}
\forall i \in \{0, \ldots, n\}, x_i = a+i\dfrac{b-a}{n}
\end{split}
\end{equation*}
\sphinxAtStartPar
Son module est \(\dfrac{b-a}{n}\)
\end{sphinxadmonition}

\begin{sphinxadmonition}{note}{Définition 2}

\sphinxAtStartPar
Si \(u\) et \(v\) sont deux subdivisions de \([a, b]\) , on dit que \(u\) est plus fine que \(v\) si tout élément de \(v\) est élément de \(u\).
\end{sphinxadmonition}

\begin{sphinxadmonition}{note}{Proposition 1}

\sphinxAtStartPar
Pour toutes subdivisions \(u\) et \(v\) de \([a, b]\) il existe une subdivision plus fine que  \(u\) et \(v\).
\end{sphinxadmonition}


\subsection{Fonctions en escalier}
\label{\detokenize{fe:fonctions-en-escalier}}
\begin{sphinxadmonition}{note}{Définition 3}

\sphinxAtStartPar
Une fonction \(\varphi\) de \([a,b]\) dans \(\mathbb{R}\) est dite en escalier si l’on peut trouver une subdivision \(u=(x_i)_{i=0}^n\) de \([a, b]\) telle que \(\varphi\) soit constante sur chacun des intervalles \(]x_{i-1}, x_i[, (1\leq i \leq n)\).
La subdivision \(u\) est dite adaptée à la fonction \(\varphi\).
\end{sphinxadmonition}

\begin{sphinxadmonition}{note}{Exemples}
\begin{itemize}
\item {} 
\sphinxAtStartPar
Une fonction constante sur l’intervalle \([a,b]\) est une fonction en escalier sur \([a,b]\).

\item {} 
\sphinxAtStartPar
La fonction \sphinxstyleemphasis{partie entière} est une fonction en escalier sur segment \([a,b]\) (pensez à une subdivision adaptée!).

\end{itemize}
\end{sphinxadmonition}

\sphinxAtStartPar
\sphinxstylestrong{Remarques}:
\begin{enumerate}
\sphinxsetlistlabels{\arabic}{enumi}{enumii}{}{.}%
\item {} 
\sphinxAtStartPar
Une fonction en escalier prend un nombre fini de valeurs. En particulier, elle est bornée.

\item {} 
\sphinxAtStartPar
Si \(u\) est une subdivision adaptée à une fonction \(\varphi\) en escalier, alors toute subdivision plus fine que \(u\) est adaptée à \(\varphi\).

\item {} 
\sphinxAtStartPar
Si \(\varphi\) et \(\psi\) sont deux fonctions en escalier sur \([a. b]\), alors il existe une subdivision adaptée à \(\varphi\) et \(\psi\).

\end{enumerate}

\begin{sphinxadmonition}{note}{Proposition 2}

\sphinxAtStartPar
L’ensemble des fonctions en escalier sur \([a, b]\) est un sous\sphinxhyphen{}espace vectoriel de l’espace des fonctions définies sur \([a, b]\)
\end{sphinxadmonition}

\begin{sphinxadmonition}{note}{Démonstration}

\sphinxAtStartPar
Soient \(\varphi\) et \(\psi\) deux fonctions en escalier sur \([a, b]\).

\sphinxAtStartPar
Soient \(\lambda\) et \(\mu\) deux réels.

\sphinxAtStartPar
D’après la proposition 1 il existe une subdivision \(u=(x_i)_{i=0}^n\) adaptées à \(\varphi\) et \(\psi\).
Donc les deux fonctions sont constantes sur \(]x_{i-1}, x_{i}[\)
Il en est de même pour \(\lambda \varphi + \mu \psi\) (constante sur \(]x_{i-1}, x_{i}[\)).
Donc, \(u\) est une subdivision adaptée pour \(\lambda \varphi + \mu \psi\) qui est par la suite une fonction en escalier.

\sphinxAtStartPar
Les autres conditions sont faciles à vérifier !
\end{sphinxadmonition}


\subsection{Intégrale d’une fonction en escalier}
\label{\detokenize{fe:integrale-d-une-fonction-en-escalier}}
\begin{sphinxadmonition}{note}{Proposition 3}

\sphinxAtStartPar
Soit \(\varphi\) une fonction en escalier sur \([a, b]\) et \(u=(x_i)_{i=0}^n\) une subdivision  adaptées à \(\varphi\). Soit \(c_i\) la valeur prise par \(\varphi\) sur \(]x_{i-1}, x_i[\) pour \(i \in \{1, \ldots, n\}\) (i.e \(\varphi(x)=c_i\) pour tout \(x \in ]x_{i-1}, x_i[\)). Alors la quantité
\begin{equation*}
\begin{split}
\sum_{i=1}^nc_i(x_i-x_{i-1})
\end{split}
\end{equation*}
\sphinxAtStartPar
ne dépend pas de la subdivision choisie.

\sphinxAtStartPar
Cette quantité s’appelle \sphinxcode{\sphinxupquote{l\textquotesingle{}intégrale}} de \(\varphi\) sur  \([a, b]\) et on le note:
\begin{equation*}
\begin{split}
\int_a^b \varphi(x)dx=\int_{[a, b]}\varphi
\end{split}
\end{equation*}\end{sphinxadmonition}

\begin{sphinxadmonition}{note}{Démonstration}

\sphinxAtStartPar
Pour toute \(u\) une subdivision adaptée à la fonction \(\varphi\), on note
\begin{equation*}
\begin{split}
I(\varphi, u)= \sum_{i=1}^nc_i(x_i-x_{i-1})
\end{split}
\end{equation*}
\sphinxAtStartPar
Le but est de montrer que \(\forall u, v\) subdivision adaptée à \(\varphi\), \(I(\varphi, u)=I(\varphi, v)\)
\begin{itemize}
\item {} 
\sphinxAtStartPar
Si \(v\) est plus fine que \(u\),
Elle est obtenue en rajoutant un nombre fini d’éléments à la subdivision \(u\). Pour démontrer que \(I(\varphi, u)=I(\varphi, v)\), il suffit de le démontrer dans le cas où  \(v\) a un élément de plus que \(u\).

\end{itemize}

\sphinxAtStartPar
Soit donc \(u=(x_i)_{i=0}^n\) et \(v=(x_1, \ldots, x_p, y, x_{p+1}, \ldots, x_n)\)

\sphinxAtStartPar
Il est claire que la fonction \(\varphi\) sur \(]x_p, y[\) et \(]y, x_{p+1}[\). Donc :
\begin{equation*}
\begin{split}
\begin{aligned}
I(\varphi, v) & =  \sum_{i=1}^{p-1}c_i(x_i-x_{i-1})+c_p(y-x_p) + c_p(x_{p+1}-y) + \sum_{i=p+2}^{n}c_i(x_i-x_{i-1})  \\ \\
 & =  \sum_{i=1}^{n}c_i(x_i-x_{i-1}) = I(\varphi, u) 
\end{aligned}
\end{split}
\end{equation*}\begin{itemize}
\item {} 
\sphinxAtStartPar
Dans le cas général:

\end{itemize}

\sphinxAtStartPar
D’après la proposition 1, il existe une subdivision \(w\) plus fine que \(u\) et \(v\), cette subdivision est aussi adaptée à \(\varphi\) (remarque 2).
Donc on aura d’une part \(I(\varphi, w)=I(\varphi, u)\) et d’autre part \(I(\varphi, w)=I(\varphi, v)\).

\sphinxAtStartPar
Par conséquent,
\begin{equation*}
\begin{split}
I(\varphi, v)=I(\varphi, u)
\end{split}
\end{equation*}\end{sphinxadmonition}


\subsection{Propriétés de l’intégrale des fonctions en escalier}
\label{\detokenize{fe:proprietes-de-l-integrale-des-fonctions-en-escalier}}
\begin{sphinxadmonition}{note}{Proposition 4}

\sphinxAtStartPar
Montrer que :

\sphinxAtStartPar
1\sphinxhyphen{} pour toutes fonctions en escalier sur \([a, b]\) \(\varphi\) et \(\psi\), et pour tous réels \(\alpha\) et \(\beta\) nous avons:
\begin{equation*}
\begin{split}
\int_{[a, b]}\alpha\varphi + \beta\psi = \alpha\int_{[a, b]}\varphi + \beta\int_{[a, b]}\psi
\end{split}
\end{equation*}
\sphinxAtStartPar
2\sphinxhyphen{} une fonction en escalier positive a une intégrale positive.

\sphinxAtStartPar
3\sphinxhyphen{} si  \(\varphi\) et \(\psi\) sont deux fonctions en escalier sur \([a, b]\) alors:
\begin{equation*}
\begin{split}
\varphi \leq \psi \Rightarrow  \int_{[a, b]}\varphi \leq \int_{[a, b]}\psi
\end{split}
\end{equation*}\end{sphinxadmonition}

\begin{sphinxadmonition}{note}{Démonstration}

\sphinxAtStartPar
1\sphinxhyphen{}

\sphinxAtStartPar
Soient \(\varphi\) et \(\psi\) deux fonctions en escalier sur \([a, b]\) et \(\alpha\) et \(\beta\) deux réels.

\sphinxAtStartPar
Soit \(u=(x_i)_{i=0}^n\) une subdivision adaptée à  \(\varphi\) et \(\psi\). Si, pour \(i \in \{1, \ldots, n\}\), \(c_i\) et \(d_i\) sont respectivement les valeurs prises par \(\varphi\) et \(\psi\) sur \(]x_{i-1}, x_i[\) alors \(\alpha\varphi + \beta\psi\) est une fonction en escalier qui prend \(\alpha c_i + \beta d_i\) sur \(]x_{i-1}, x_i[\).

\sphinxAtStartPar
Nous avons donc,
\begin{equation*}
\begin{split}
\begin{aligned}
\int_{[a, b]}\alpha\varphi + \beta\psi & = \sum_{i=1}^{n}(\alpha c_i + \beta d_i)(x_i-x_{i-1})   \\ \\
 & =  \alpha\sum_{i=1}^{n} c_i(x_i-x_{i-1}) +  \alpha\sum_{i=1}^{n} d_i(x_i-x_{i-1}) \\ \\
 & = \alpha\int_{[a, b]}\varphi + \beta\int_{[a, b]}\psi
\end{aligned}
\end{split}
\end{equation*}
\sphinxAtStartPar
2\sphinxhyphen{}

\sphinxAtStartPar
Soit \(\varphi\) une fonction en escalier sur \([a, b]\) qui est positive et  \(u=(x_i)_{i=0}^n\) une subdivision adaptée à \(\varphi\). Puisqu’elle est positive, les valeurs \(c_i\) prise par \(\varphi\) sont positives. D’autre part, puisque pour tout \(i \in \{1, \ldots, n\}, x_{i-1} \leq x_i\) (par définition de la subdivision) alors \(\int_{[a, b]}\varphi = \sum_{i=1}^{n} c_i(x_i-x_{i-1}) \geq 0\).

\sphinxAtStartPar
3\sphinxhyphen{}

\sphinxAtStartPar
La fonction \(\varphi - \psi\) est une fonction en escalier positive donc son intégral est positive.
\end{sphinxadmonition}

\begin{sphinxadmonition}{note}{Proposition 5}

\sphinxAtStartPar
Une fonction \(\varphi\) est en escalier sur \([a, b]\) si et seulement si pour tout \(c \in ]a, b[\), ses restrictions sur \(]a, c[\) et \(]c, b[\) le sont. Le cas échéant,
\begin{equation*}
\begin{split}
\int_{[a, b]}\varphi = \int_{[a, c]}\varphi_{|[a, c]} + \int_{[c, b]}\varphi_{|[c, b]}
\end{split}
\end{equation*}\end{sphinxadmonition}

\begin{sphinxadmonition}{note}{Démonstration}
\begin{itemize}
\item {} 
\sphinxAtStartPar
\(\Rightarrow\)
Soit \(\varphi\) une fonction en escalier sur \([a, b]\) et \(u\) une subdivision adaptée à \(\varphi\). On ajoutant \(c\) a la subdivision \(u\) on obtient un subdivision \(v=(x_i)_{i=0}^n\) qui est encore adaptée à \(\varphi\). Soit \(p\) l’entier naturel tel que \(c=x_p\). Alors nous avons :
\begin{itemize}
\item {} 
\sphinxAtStartPar
\((x_1, \ldots, x_p)\) une subdivision de  \([a, c]\) et puisque \(\varphi\) est constante sur chaque \(]x_{i-1}, x_{i}[\) donc \(\varphi_{|[a, c]}\) est fonction en escalier sur \([a, c]\). Nous avons donc :

\end{itemize}
\begin{equation*}
\begin{split}
    \int_{[a, c]}\varphi_{|[a, c]} = \sum_{i=1}^{p} c_i(x_i-x_{i-1})
    \end{split}
\end{equation*}\begin{itemize}
\item {} 
\sphinxAtStartPar
\((x_{p+1}, \ldots, x_n)\) une subdivision de  \([c, b]\) et puisque \(\varphi\) est constante sur chaque \(]x_{i-1}, x_{i}[\) donc \(\varphi_{|[c, b]}\) est fonction en escalier sur \([c, b]\). Nous avons donc :

\end{itemize}
\begin{equation*}
\begin{split}
    \int_{[c, b]}\varphi_{|[c, b]} = \sum_{i=p+1}^{n} c_i(x_i-x_{i-1})
    \end{split}
\end{equation*}
\sphinxAtStartPar
Par suite :
\begin{equation*}
\begin{split}
    \begin{aligned}
    \int_{[a, b]}\varphi &= \sum_{i=1}^{n} c_i(x_i-x_{i-1})=\sum_{i=1}^{p} c_i(x_i-x_{i-1}) + \sum_{i=p+1}^{n} c_i(x_i-x_{i-1}) \\ \\ 
     &= \int_{[a, c]}\varphi_{|[a, c]} + \int_{[c, b]}\varphi_{|[c, b]}
    \end{aligned}
    \end{split}
\end{equation*}
\end{itemize}
\begin{itemize}
\item {} 
\sphinxAtStartPar
\(\Leftarrow\)
Supposons que \(\varphi_{|[a, c]}\) et \(\varphi_{|[c, b]}\) sont en escalier sur \([a, c]\) et \([c, b]\) respectivement.

\end{itemize}

\sphinxAtStartPar
Soit \(u=(x_i)_{i=0}^p\) (respectivement \(v= (x_i)_{i=0}^q\)) une subdivision de \([a, c]\) (respectivement de \([c, b]\)) adaptée à \(\varphi_{|[a, c]}\) (respectivement q \(\varphi_{|[c, b]}\)).

\sphinxAtStartPar
Alors, \((x_0, \ldots, x_{p-1}, x_p=c=y_1, \ldots, y_q)\) est une subdivision \([a, b]\). De plus \(\varphi\) est constante sur chaque intervalle \(]x_{i-1}, x_{i}[\) et \(]y_{i-1}, y_{i}[\). Donc \(\varphi\) est en escalier sur \([a, b]\).
\end{sphinxadmonition}


\section{Fonctions continues par morceaux}
\label{\detokenize{fcm:fonctions-continues-par-morceaux}}\label{\detokenize{fcm::doc}}

\subsection{Définition, exemples}
\label{\detokenize{fcm:definition-exemples}}
\begin{sphinxadmonition}{note}{Définition 4}

\sphinxAtStartPar
Une application \(f\) de \([a, b]\) dans \(\mathbb{R}\) est dite continue par morceaux s’il existe une subdivision \(u=(x_i)_{i=0}^n\) de \([a, b]\) telle que pour chaque \(i \in \{1,\ldots, n\}\) la restriction de \(f\) à \(]x_{i-1}, x_i[\) soit continue et admette des limites finies en \(x_{i-1}\) et \(x_{i}\).
\end{sphinxadmonition}

\sphinxAtStartPar
La subdivision \(u\) est dite adaptée à la fonction \(f\).

\sphinxAtStartPar
L’exemple suivant donne une illustration graphique d’une fonction continue par morceaux.

\sphinxAtStartPar
\sphinxstylestrong{Illustration avec un exemple graphique}:

\noindent{\hspace*{\fill}\sphinxincludegraphics[width=500\sphinxpxdimen]{{fig1}.png}\hspace*{\fill}}

\begin{sphinxadmonition}{note}{Exemples}
\begin{itemize}
\item {} 
\sphinxAtStartPar
Toute fonction en escalier est continue par morceaux.

\item {} 
\sphinxAtStartPar
Toute fonction continue est continue par morceaux.

\item {} 
\sphinxAtStartPar
La fonction \(f\) définie sur \([-1, 1]\) par :

\end{itemize}
\begin{equation*}
\begin{split}f(0)=0 \mbox{  et  } \forall x \in [-1, 1] \setminus \{0\}, f(x)=\dfrac{1}{x}
\end{split}
\end{equation*}
\sphinxAtStartPar
n’est pas continue par morceaux, car elle n’a pas de limite finie à droite et à gauche de \(0\).
\end{sphinxadmonition}

\begin{sphinxadmonition}{note}{Remarques}

\sphinxAtStartPar
Comme pour les fonctions en escalier, on peut vérifier que :
\begin{itemize}
\item {} 
\sphinxAtStartPar
si \(u\) est une subdivision adaptée à une fonction \(f\) continue par marceaux,
alors toute subdivision plus fine que \(u\) est adaptée à \(f\),

\item {} 
\sphinxAtStartPar
si \(f\) et \(g\) sont deux fonctions continues par morceaux sur \([a, b]\) , alors il existe
une subdivision adaptée à \(f\) et \(g\).

\end{itemize}
\end{sphinxadmonition}

\begin{sphinxadmonition}{note}{Proposition 6}

\sphinxAtStartPar
Une fonction continue par morceaux sur  \([a, b]\) est bornée sur \([a, b]\) .
\end{sphinxadmonition}

\begin{sphinxadmonition}{note}{Démonstration}

\sphinxAtStartPar
Soit \(f\) une fonction continue par morceaux sur  \([a, b]\). Soit \(u=(x_i)_{i=0}^n\) une subdivision adaptée à \(f\).

\sphinxAtStartPar
Pour chaque \(i \in \{1, \ldots, n\}\) admet des limites finies en \(x_{i-1}\) et \(x_i\). Donc la restriction de \(f\) sur \(]x_{i-1}, x_i[\) admet un prolongement par continuité sur \([x_{i-1}, x_i]\) qui donc borne. Par suite, \(f\) est bornée sur \(]x_{i-1}, x_i[\). Soit \(M_i = \sup_{]x_{i-1}, x_i[}|f|\)

\sphinxAtStartPar
En prenant \(M = max (M_1, M_2, \ldots, M_n, |f(x_0)|, \ldots, |f(x_0)|)\), nous aurons \(\forall x \in [a, b], |f(x)| \leq M\).

\sphinxAtStartPar
Par suite, \(f\) est bornée sur \([a, b]\).
\end{sphinxadmonition}

\begin{sphinxadmonition}{note}{Proposition 7}

\sphinxAtStartPar
Soient \(f\) et \(g\) deux fonctions continues par morceaux sur \([a,b]\). Les assertions suivantes sont correctes :
\begin{itemize}
\item {} 
\sphinxAtStartPar
\( \forall \lambda, \mu \in \mathbb R, \lambda f + \mu g\) est continue par morceaux.

\item {} 
\sphinxAtStartPar
\(fg\) est continue par morceaux.

\end{itemize}
\end{sphinxadmonition}

\begin{sphinxadmonition}{note}{Démonstration}

\sphinxAtStartPar
Soit \(u=(x_i)_{i=0}^n\) une subdivision adaptée à \(f\) et \(g\).

\sphinxAtStartPar
Les restrictions des fonctions \(f\) et \(g\) à chacun des intervalles \( ]x_{i-1}, x_i[\) sont continues et admettent des limites finies en  \(x_{i-1}\) et \(x_i\), donc il en est de même pour \(\lambda f + \mu g\) et \(fg\). Les fonctions
\(\lambda f + \mu g\) et \(fg\) sont donc continues par morceaux sur \([a, b]\).
\end{sphinxadmonition}


\subsection{Intégrale d’une fonction continue par morceaux}
\label{\detokenize{fcm:integrale-d-une-fonction-continue-par-morceaux}}
\begin{sphinxadmonition}{note}{Théorème (admis)}

\sphinxAtStartPar
Soit \(f\) une fonction continue par morceaux sur le segment \([a, b]\) . Pour tout
réel \(\epsilon > O\):
\begin{itemize}
\item {} 
\sphinxAtStartPar
il existe une fonction en escalier \(\theta\) telle que \(|f - \theta| \leq \epsilon\)

\item {} 
\sphinxAtStartPar
il existe des fonctions en escalier \(\varphi\) et \(\psi\) telles que :

\end{itemize}
\begin{equation*}
\begin{split}
\varphi \leq f \leq \psi \mbox{   et  } \psi - \varphi \leq \epsilon
\end{split}
\end{equation*}\end{sphinxadmonition}

\sphinxAtStartPar
\sphinxstylestrong{Notations}: Dans ce qui suit, pour une fonction continue par morceaux \(f\) nous allons adopte les notations suivantes :
\begin{itemize}
\item {} 
\sphinxAtStartPar
\(\mathcal{E}^+(f)\) l’ensemble des fonctions en escalier plus grandes que \(f\).

\item {} 
\sphinxAtStartPar
\(\mathcal{E}^-(f)\) l’ensemble des fonctions en escalier plus petites que \(f\).

\end{itemize}

\begin{sphinxadmonition}{note}{Proposition 8}

\sphinxAtStartPar
Soit \(f\) une fonction continue par morceaux sur le segment \([a, b]\). Alors :
\begin{itemize}
\item {} 
\sphinxAtStartPar
\(\left\{\int_{[a, b]} \varphi | \varphi \in \mathcal{E}^-(f)\right\}\) admet une borne supérieure,

\item {} 
\sphinxAtStartPar
\(\left\{\int_{[a, b]} \psi | \psi \in \mathcal{E}^+(f)\right\}\) admet une borne inferieure,

\end{itemize}

\sphinxAtStartPar
de plus,
\begin{equation*}
\begin{split}
sup\left\{\int_{[a, b]} \varphi | \varphi \in \mathcal{E}^-(f)\right\}= inf\left\{\int_{[a, b]} \psi | \psi \in \mathcal{E}^+(f)\right\}
\end{split}
\end{equation*}\end{sphinxadmonition}

\begin{sphinxadmonition}{note}{Démonstration}

\sphinxAtStartPar
Soit \(f\) une fonction continue par morceaux sur \([a, b]\). Donc \(\left\{f(x)|x\in [a, b]\right\}\) est une partie non vide de \(\mathbb R\) bornée donc admet une borne supérieure et une borne inferieure. Soit \(m=\inf_{[a, b]} f\) et \(M=\sup_{[a, b]} f\).
\begin{itemize}
\item {} 
\sphinxAtStartPar
Les deux fonctions constantes \(m\) et \(M\) sur \([a, b]\) sont aussi continues par morceaux sur \([a, b]\). \(\left\{\int_{[a, b]} \varphi | \varphi \in \mathcal{E}^-(f)\right\}\) est donc une partie de \(\mathbb R\) non vide majorée (par \(M(b-a)\)). Alors, elle possède une borne supérieure (\(\alpha\)). De même, \(\left\{\int_{[a, b]} \psi | \psi \in \mathcal{E}^+(f)\right\}\) une partie de \(\mathbb R\) non vide minorée donc possède une borne inferieure (\(\beta\)).

\item {} 
\sphinxAtStartPar
Toute fonction \(\varphi \in \mathcal{E}^-(f)\) est inférieure à toute fonction \(\psi \in \mathcal{E}^+(f)\). Par suite :

\end{itemize}
\begin{equation*}
\begin{split}
\int_{[a, b]} \varphi \leq \int_{[a, b]} \psi
\end{split}
\end{equation*}
\sphinxAtStartPar
Fixons \(\psi \in \mathcal{E}^+(f)\). L’ensemble \(\left\{\int_{[a, b]} \varphi | \varphi \in \mathcal{E}^-(f)\right\}\) est majoreé par \(\int_{[a, b]} \psi\). Alors nous avons forcement \(\alpha \leq \int_{[a, b]} \psi\) et ça pour tout \(\psi \in \mathcal{E}^+(f)\). Donc \(\alpha\) est un minorant de \(\left\{\int_{[a, b]} \psi | \psi \in \mathcal{E}^+(f)\right\}\). Par conséquent, \(\alpha \leq \beta\).

\sphinxAtStartPar
Donc
\begin{equation*}
\begin{split}
\alpha \leq \beta
\end{split}
\end{equation*}\begin{itemize}
\item {} 
\sphinxAtStartPar
soit \(\epsilon >0\). En utilisant le théorème précédant, il existe deux fonctions en escalier \(\varphi \in \mathcal{E}^-(f)\) et \(\psi \in \mathcal{E}^+(f)\)  tel que \(\psi - \varphi \leq \epsilon\).

\end{itemize}

\sphinxAtStartPar
Donc \(\int_{[a, b]} \psi - \int_{[a, b]} \varphi \leq \int_{[a, b]} \epsilon = \epsilon(b-a)\)

\sphinxAtStartPar
donc \(0 \leq \beta - \alpha \leq \epsilon(b-a)\)

\sphinxAtStartPar
Et ca pour tout \(\epsilon \geq 0\).

\sphinxAtStartPar
Donc \(\alpha = \beta\).
\end{sphinxadmonition}

\begin{sphinxadmonition}{note}{Définition}

\sphinxAtStartPar
Soit \(f\) une fonction continue par morceaux sur le segment \([a, b]\).

\sphinxAtStartPar
On appelle intégrale de \(f\) sur \([a, b]\) le réel
\begin{equation*}
\begin{split}
\int_{[a, b]} f = sup\left\{\int_{[a, b]} \varphi | \varphi \in \mathcal{E}^-(f)\right\}= inf\left\{\int_{[a, b]} \psi | \psi \in \mathcal{E}^+(f)\right\}
\end{split}
\end{equation*}\end{sphinxadmonition}

\sphinxAtStartPar
\sphinxstylestrong{Question} : Comparer l’intégrale d’une fonction en escalier avec son intégral en tant que fonction continue par morceaux.

\begin{sphinxadmonition}{note}{Indications pour la réponse}
\begin{itemize}
\item {} 
\sphinxAtStartPar
Une fonction en escalier est continue par morceaux ;

\item {} 
\sphinxAtStartPar
si \(f\) est une fonction en escalier, alors \(f \in \left\{\int_{[a, b]} \varphi | \varphi \in \mathcal{E}^-(f)\right\}\) et \( f \in \left\{\int_{[a, b]} \psi | \psi \in \mathcal{E}^+(f)\right\}\).

\end{itemize}
\end{sphinxadmonition}

\sphinxAtStartPar
Nous avons vu que si \(f\) une fonction continue par morceaux sur \([a, b]\), alors \(f\) est bornée. Soient \(m = inf \left\{f(x)| x\in [a, b]\right\}\) et \(M = sup \left\{f(x)| x\in [a, b]\right\}\). \(m\) et \(M\) sont donc des fonctions en escalier sur \([a, b]\). Leurs intégrales sont respectivement \(m(b-a)\) et \(M(b-a)\).

\sphinxAtStartPar
Et puisque \(m \leq f \leq M\), nous avons \(m(b-a)\leq \int_{[a,b]} f \leq M(b-a)\).

\sphinxAtStartPar
Donc, la quantité \(\dfrac{1}{b-a}\int_{[a, b]} f\) est comprise entre \(m\) et \(M\).

\begin{sphinxadmonition}{note}{Définition}

\sphinxAtStartPar
Soit \(f\) une fonction continue par morceaux sur \([a, b]\). La quantité \(\dfrac{1}{b-a}\int_{[a, b]} f\) s’appelle \sphinxstylestrong{la valeur moyenne} de \(f\).
\end{sphinxadmonition}


\section{Propriétés de l’intégrale}
\label{\detokenize{pptint:proprietes-de-l-integrale}}\label{\detokenize{pptint::doc}}
\begin{sphinxadmonition}{note}{Proposition (linéarité)}

\sphinxAtStartPar
Soient \(f_1, f_2\) deux fonctions continues par morceaux sur \([a, b]\), \(\lambda_1, \lambda_2 \in \mathbb R\). Alors :
\begin{equation*}
\begin{split}
\int_{[a, b]} (\lambda_1f_1 + \lambda_2f_2) = \lambda_1 \int_{[a, b]} f_1 + \lambda_2 \int_{[a, b]} f_2
\end{split}
\end{equation*}\end{sphinxadmonition}

\sphinxAtStartPar
On dit que l’intégrale est une forme linéaire sur l’espace vectoriel des fonctions continues par morceaux sur \([a, b]\).

\begin{sphinxadmonition}{note}{Démonstration}
\begin{itemize}
\item {} 
\sphinxAtStartPar
Soient \(\epsilon>0\) et \(f\) est une fonction continue par morceaux, alors il existe \(\theta\) est fonction en escalier telle que \(|f-\theta|\leq \epsilon\). On a \(\theta - \epsilon \leq f \leq \theta + \epsilon\). Alors

\end{itemize}
\begin{equation*}
\begin{split}
 \int_{[a, b]} (\theta - \epsilon) \leq  \int_{[a, b]} f \leq  \int_{[a, b]} (\theta + \epsilon)
 \end{split}
\end{equation*}
\sphinxAtStartPar
et donc
\begin{equation*}
\begin{split}
\left|\int_{[a, b]} f- \int_{[a, b]} \theta \right| \leq (b-a)\epsilon
\end{split}
\end{equation*}\begin{itemize}
\item {} 
\sphinxAtStartPar
Maintenant, soient \(f_1\) et \(f_2\) deux fonctions continues par morceaux sur \([a, b]\), ainsi que \(\lambda_1, \lambda_2\) deux réels.

\end{itemize}

\sphinxAtStartPar
Pour \(\epsilon>0\) quelconque, il existe \(\theta_1, \theta_2\) deux fonctions en escalier telles que :
\begin{equation*}
\begin{split}
|f_1 - \theta_1| \leq \epsilon ~~~ \mbox{ et } ~~~ |f_2 - \theta_2| \leq \epsilon
\end{split}
\end{equation*}
\sphinxAtStartPar
Ce qui entraine
\begin{equation*}
\begin{split}
\left|\int_{[a, b]} f_1- \int_{[a, b]} \theta_1 \right| \leq (b-a)\epsilon
\end{split}
\end{equation*}
\sphinxAtStartPar
et
\begin{equation*}
\begin{split}
\left|\int_{[a, b]} f_2- \int_{[a, b]} \theta_2 \right| \leq (b-a)\epsilon
\end{split}
\end{equation*}
\sphinxAtStartPar
Donc
\begin{equation*}
\begin{split}
|\lambda_1|\left|\int_{[a, b]} f_1- \int_{[a, b]} \theta_1 \right| \leq |\lambda_1|(b-a)\epsilon
\end{split}
\end{equation*}
\sphinxAtStartPar
et
\begin{equation*}
\begin{split}
|\lambda_2|\left|\int_{[a, b]} f_2- \int_{[a, b]} \theta_2 \right| \leq |\lambda_2|(b-a)\epsilon
\end{split}
\end{equation*}
\sphinxAtStartPar
Donc (puisque l’intégrale sur les fonctions en escalier est linéaire)
\begin{equation*}
\begin{split}
\left|\lambda_1\int_{[a, b]} f_1- \int_{[a, b]} \lambda_1\theta_1 \right| \leq |\lambda_1|(b-a)\epsilon
\end{split}
\end{equation*}
\sphinxAtStartPar
et
\begin{equation*}
\begin{split}
\left|\lambda_2\int_{[a, b]} f_2- \int_{[a, b]} \lambda_2\theta_2 \right| \leq |\lambda_2|(b-a)\epsilon
\end{split}
\end{equation*}
\sphinxAtStartPar
Par suite
\begin{equation*}
\begin{split}
\left|\lambda_1\int_{[a, b]} f_1 +\lambda_2\int_{[a, b]} f_2 - \left(\int_{[a, b]} \lambda_1\theta_1 + \int_{[a, b]} \lambda_2\theta_2\right)\right| \leq (|\lambda_1|+|\lambda_2|)(b-a)\epsilon
\end{split}
\end{equation*}
\sphinxAtStartPar
On pose
\begin{equation*}
\begin{split}
f= \lambda_1 f_1 + \lambda_2 f_2 ~~ \mbox{ et } \theta = \lambda_1 \theta_1 + \lambda_2 \theta_2
\end{split}
\end{equation*}
\sphinxAtStartPar
On a
\begin{equation*}
\begin{split}
|f-\theta| \leq |\lambda_1||f_1-\theta_1| + |\lambda_2||f_2 - \theta_2| \leq (|\lambda_1|+|\lambda_2|)\epsilon
\end{split}
\end{equation*}
\sphinxAtStartPar
Et par suite
\begin{equation*}
\begin{split}
\left|\int_{[a, b]} f- \int_{[a, b]} \theta \right| \leq (b-a)(|\lambda_1|+|\lambda_2|)\epsilon
\end{split}
\end{equation*}
\sphinxAtStartPar
On pose
\begin{equation*}
\begin{split}
I = \int_{[a, b]} \theta = \int_{[a, b]} \lambda_1\theta_1 + \int_{[a, b]} \lambda_2\theta_2
\end{split}
\end{equation*}
\sphinxAtStartPar
et
\begin{equation*}
\begin{split}
\Delta = \left|\int_{[a, b]} f- \left(\lambda_1\int_{[a, b]} f_1 +\lambda_2\int_{[a, b]} f_2\right)\right|
\end{split}
\end{equation*}
\sphinxAtStartPar
Alors
\begin{equation*}
\begin{split}
\begin{aligned}
\Delta &=& \left|\int_{[a, b]} f- I + I - \left(\lambda_1\int_{[a, b]} f_1 +\lambda_2\int_{[a, b]} f_2\right)\right| \\ \\
& \leq & \left|\int_{[a, b]} f- I\right| + \left| I - \left(\lambda_1\int_{[a, b]} f_1 +\lambda_2\int_{[a, b]} f_2\right)\right| \\ \\
& \leq & 2(b-a)(|\lambda_1|+|\lambda_2|)\epsilon
\end{aligned}
\end{split}
\end{equation*}
\sphinxAtStartPar
On en déduit
\begin{equation*}
\begin{split}
\forall \epsilon >0, \Delta \leq 2(b-a)(|\lambda_1|+|\lambda_2|)\epsilon
\end{split}
\end{equation*}
\sphinxAtStartPar
Ce qui prouve que \(\Delta =0\) et donc
\begin{equation*}
\begin{split}
 \left|\int_{[a, b]} f = \lambda_1\int_{[a, b]} f_1 +\lambda_2\int_{[a, b]} f_2\right|
\end{split}
\end{equation*}\end{sphinxadmonition}

\sphinxAtStartPar
Deux fonctions continues par morceaux sur l’intervalle \([a. b]\)  qui sont égales sauf en un nombre fini de points ont la même intégrale car leur différence, qui est nulle sauf
en un nombre fini de points, est une fonction en escalier dont l’intégrale est nulle.

\begin{sphinxadmonition}{note}{Proposition (relation de Chasles)}

\sphinxAtStartPar
Soit \(c \in [a, b]\) et \(f\) une fonction définie sur \([a, b]\).

\sphinxAtStartPar
La fonction \(f\) est continue par morceaux sur \([a, b]\) si, et seulement si, ses
restrictions à \([a, c]\) et à \([c, b]\) sont continues par morceaux, et l’on a alors :
\begin{equation*}
\begin{split}
\int_{[a, b]} f = \int_{[a, c]} f_{|[a, c]}  + \int_{[c, b]} f_{|[c, b]}
\end{split}
\end{equation*}\end{sphinxadmonition}

\begin{sphinxadmonition}{note}{Démonstration}

\sphinxAtStartPar
Soit \(\varphi \in \mathcal E^-(f)\) (une fonction en escalier plus petite que \(f\)).

\sphinxAtStartPar
On a \(\varphi_{|[a, c]} \in \mathcal E^-(f_{|[a, c]})\) et \(\varphi_{|[c, b]} \in \mathcal E^-(f_{|[c, b]})\).

\sphinxAtStartPar
Par suite
\begin{equation*}
\begin{split}
\begin{aligned}
\int_{[a, b]}\varphi &=& \int_{[a, c]} \varphi_{|[a, c]} + \int_{[c, b]} \varphi_{|[c, b]} \\ \\
&\leq& \int_{[a, c]} f_{|[a, c]} + \int_{[c, b]} f_{|[c, b]}
\end{aligned}
\end{split}
\end{equation*}
\sphinxAtStartPar
Le réel \(\int_{[a, c]} f_{|[a, c]} + \int_{[c, b]} f_{|[c, b]}\) est un majorant de de \(\left\{\int_{[a, b]}\varphi ~~|~~ \varphi \in \mathcal E^-(f) \right\}\).

\sphinxAtStartPar
Donc il est plus grands que sa borne supérieure (\(\int_{[a, b]} f\)).
Ce qui donne
\begin{equation*}
\begin{split}
\int_{[a, b]} f \leq \int_{[a, c]} f_{|[a, c]} + \int_{[c, b]} f_{|[c, b]}
\end{split}
\end{equation*}
\sphinxAtStartPar
En applique les mêmes étapes pour \(-f\)

\sphinxAtStartPar
Soit \(\varphi \in \mathcal E^-(-f)\) (une fonction en escalier plus petite que \(-f\)).

\sphinxAtStartPar
On a \(\varphi_{|[a, c]} \in \mathcal E^-(-f_{|[a, c]})\) et \(\varphi_{|[c, b]} \in \mathcal E^-(-f_{|[c, b]})\).

\sphinxAtStartPar
Par suite
\begin{equation*}
\begin{split}
\begin{aligned}
\int_{[a, b]}\varphi &=& \int_{[a, c]} \varphi_{|[a, c]} + \int_{[c, b]} \varphi_{|[c, b]} \\ \\
&\leq& \int_{[a, c]} -f_{|[a, c]} + \int_{[c, b]} -f_{|[c, b]}
\end{aligned}
\end{split}
\end{equation*}
\sphinxAtStartPar
Le reel \(\int_{[a, c]} -f_{|[a, c]} + \int_{[c, b]} -f_{|[c, b]}\) est un majorant de de \(\left\{\int_{[a, b]}\varphi ~~|~~ \varphi \in \mathcal E^-(-f) \right\}\).

\sphinxAtStartPar
Donc il est plus grands que sa borne supérieure (\(\int_{[a, b]} -f\)).
Ce qui donne
\begin{equation*}
\begin{split}
\int_{[a, b]} -f \leq \int_{[a, c]} -f_{|[a, c]} + \int_{[c, b]} -f_{|[c, b]}
\end{split}
\end{equation*}
\sphinxAtStartPar
Et d’après la linéarité de l’intégrale nous avons
\begin{equation*}
\begin{split}
\int_{[a, b]} f \geq \int_{[a, c]} f_{|[a, c]} + \int_{[c, b]} f_{|[c, b]}
\end{split}
\end{equation*}
\sphinxAtStartPar
En fin
\begin{equation*}
\begin{split}
\int_{[a, b]} f = \int_{[a, c]} f_{|[a, c]} + \int_{[c, b]} f_{|[c, b]}
\end{split}
\end{equation*}\end{sphinxadmonition}

\sphinxAtStartPar
\sphinxstylestrong{Remarque}:

\sphinxAtStartPar
Soient \(f\) une fonction continue par morceaux sur \([a, b]\), et \(u=(x_i)_{i\in\{1,\ldots,n\}}\) une subdivision adaptée à \(f\). Pour \(i \in \{1,\ldots,n\}\), notons \(f_i\) la fonction continue sur \([x_{i-1}, x_i]\) tel que \(\forall x \in ]x_{i-1}, x_i[, f_i(x) = f(x)\). Alors d’après la relation de Chasles :
\begin{equation*}
\begin{split}
\int_{[a, b]} f = \sum_{i=1}^n \int_{[x_{i-1}, x_i]}f = \sum_{i=1}^n \int_{[x_{i-1}, x_i]}f_i
\end{split}
\end{equation*}

\subsection{Quelques inégalités}
\label{\detokenize{pptint:quelques-inegalites}}
\begin{sphinxadmonition}{note}{Proposition}
\begin{itemize}
\item {} 
\sphinxAtStartPar
Une fonction positive et continue par morceaux à une intégrale positive.

\item {} 
\sphinxAtStartPar
Si \(f\) et \(g\) sont deux fonctions continues par morceaux sur \([a, b]\) alors:

\end{itemize}
\begin{equation*}
\begin{split}
f \leq g \Rightarrow \int_{[a, b]}f \leq \int_{[a, b]}g 
\end{split}
\end{equation*}\end{sphinxadmonition}

\begin{sphinxadmonition}{note}{Démonstration}
\begin{itemize}
\item {} 
\sphinxAtStartPar
Si \(f\) est une fonction continue par morceaux et positive alors la fonction nulle (qui constante donc en escalier) appartient à \(\mathcal E^-(f)\). Donc son intégrale (qui vaut 0) est inférieure à l’intégrale de \(f\). Par suite \(\int_{[a, b]} f \geq 0\).

\item {} 
\sphinxAtStartPar
On applique le résultat précèdent à \(g-f\) puis on utilise la linéarité de l’intégrale.

\end{itemize}
\end{sphinxadmonition}

\begin{sphinxadmonition}{note}{Théorème}

\sphinxAtStartPar
Si \(f\) est continue par morceaux sur \([a, b]\), alors \(|f|\) est continue par morceaux sur \([a, b]\) et:
\begin{equation*}
\begin{split}
\left |\int_{[a, b]}f \right | \leq \int_{[a, b]}|f|
\end{split}
\end{equation*}\end{sphinxadmonition}

\begin{sphinxadmonition}{note}{Démonstration}

\sphinxAtStartPar
Soient \(f\) une fonction continue par morceaux sur \([a, b]\) et \(u=(x_i)_{i=0}^n\) une subdivision adaptée à \(f\). Alors, pour tout \(i \in \{1, \ldots, n\}\) la restriction de \(f\) sur chacun des intervalles \(]x_{i-1}, x_i[\) est continue et admet des limites finies en \(x_{i-1}\) et \(x_i\). Il en est de même pour \(|f|\) d’après les propriétés des limites. Donc \(|f|\) est une fonction continue par morceaux sur \([a, b]\).

\sphinxAtStartPar
Nous avons \(-|f| \leq f \leq |f|\) donc
\begin{equation*}
\begin{split}
- \int_{[a, b]}|f| \leq \int_{[a, b]}f \leq \int_{[a, b]}|f|  
\end{split}
\end{equation*}
\sphinxAtStartPar
Ce qui veut dire
\begin{equation*}
\begin{split}
\left| \int_{[a, b]}f \right | \leq \int_{[a, b]}|f|  
\end{split}
\end{equation*}\end{sphinxadmonition}

\begin{sphinxadmonition}{note}{Proposition (Inégalité de la moyenne)}

\sphinxAtStartPar
Si \(f\) et \(g\) sont deux fonctions continues par morceaux sur \([a, b]\), alors:
\begin{equation*}
\begin{split}
\left |\int_{[a, b]}fg \right | \leq \sup_{[a, b]} |f| \int_{[a, b]}|g|
\end{split}
\end{equation*}\end{sphinxadmonition}

\begin{sphinxadmonition}{note}{Démonstration}

\sphinxAtStartPar
Soit \(M = sup_{[a,b]}|f|\).
Nous avons
\begin{equation*}
\begin{split}
\forall x \in [a, b], |f(x)g(x)|=|f(x)||g(x)| \leq M |g(x)|
\end{split}
\end{equation*}
\sphinxAtStartPar
Donc
\begin{equation*}
\begin{split}
\left |\int_{[a, b]}fg \right | \leq \int_{[a, b]}|fg| \leq \int_{[a, b]}M|g| =M\int_{[a, b]}|g|
\end{split}
\end{equation*}\end{sphinxadmonition}

\begin{sphinxadmonition}{note}{Corollaire}

\sphinxAtStartPar
Si \(f\) est une fonction continue par morceaux sur \([a, b]\), alors:
\begin{equation*}
\begin{split}
\left |\int_{[a, b]}f \right | \leq (b-a)  \sup_{[a, b]}|f|
\end{split}
\end{equation*}\end{sphinxadmonition}

\begin{sphinxadmonition}{note}{Démonstration}

\sphinxAtStartPar
En pose \(g=1\) est on applique l’inégalité de la moyenne.
\end{sphinxadmonition}

\begin{sphinxadmonition}{note}{Théorème (Inégalité de Cauchy\sphinxhyphen{}Schwarz)}

\sphinxAtStartPar
Si \(f\) et \(g\) sont deux fonctions continues par morceaux sur \([a, b]\), alors:
\begin{equation*}
\begin{split}
\left (\int_{[a, b]}fg \right ) ^2 \leq \int_{[a, b]} f^2 \int_{[a, b]}g^2
\end{split}
\end{equation*}\end{sphinxadmonition}

\begin{sphinxadmonition}{note}{Démonstration}

\sphinxAtStartPar
On pose
\begin{equation*}
\begin{split}
P(\lambda) = \int_{[a, b]} (f+\lambda g)^2 = \lambda^2 \int_{[a, b]} g^2 + 2\lambda \int_{[a, b]} fg + \int_{[a, b]} f^2
\end{split}
\end{equation*}
\sphinxAtStartPar
\(P\) est donc une fonction polynomiale de degré au plus égale à 2 qui est positive pour tout \(\lambda \in \mathbb R\).
\begin{itemize}
\item {} 
\sphinxAtStartPar
Si \(\int_{[a, b]} g^2 = 0\), alors la fonction \(P\) ne peut pas changer de signe (car elle est positive) donc ne peut pas être de degré 1. On en déduit \(\int_{[a, b]} fg = 0\)

\end{itemize}

\sphinxAtStartPar
Donc \(\left (\int_{[a, b]} fg\right)^2 = 0=\int_{[a, b]} f^2 \int_{[a, b]} g^2\)
\begin{itemize}
\item {} 
\sphinxAtStartPar
Sinon, \(P\) est polynôme de degré 2 qui positif pour tout \(\lambda \in \mathbb R\). Son discriminent

\end{itemize}
\begin{equation*}
\begin{split}
\Delta = 4 \left (\left (\int_{[a, b]} fg\right)^2- \int_{[a, b]} f^2 \int_{[a, b]} g^2 \right)
\end{split}
\end{equation*}
\sphinxAtStartPar
est donc negatif. Ce donne
\begin{equation*}
\begin{split}
\left (\int_{[a, b]} fg\right)^2 \leq \int_{[a, b]} f^2 \int_{[a, b]} g^2
\end{split}
\end{equation*}\end{sphinxadmonition}

\sphinxAtStartPar
On peut écrire l’inégalité de Cauchy\sphinxhyphen{}Schwarz comme suit :
\begin{equation*}
\begin{split}
\left |\int_{[a, b]}fg \right | \leq \left (\int_{[a, b]} f^2 \right)^{\frac{1}{2}}  \left (\int_{[a, b]}g^2\right)^{\frac{1}{2}} 
\end{split}
\end{equation*}

\subsection{Cas des fonctions continues}
\label{\detokenize{pptint:cas-des-fonctions-continues}}
\begin{sphinxadmonition}{note}{Théorème}

\sphinxAtStartPar
Une fonction \sphinxstylestrong{continue et positive} sur \([a, b]\) est nulle si, et seulement si, son intégrale sur \([a, b]\) est nulle.
\end{sphinxadmonition}

\begin{sphinxadmonition}{note}{Démonstration}

\sphinxAtStartPar
Soit \(f\) une fonction continue et positive sur \([a, b]\).

\sphinxAtStartPar
Si \(f\) est nulle alors son intégrale est nulle.

\sphinxAtStartPar
Montrons maintenant, que si l’intégrale de \(f\) est nulle alors \(f\) est la fonction nulle.

\sphinxAtStartPar
Par absurde, on suppose que \(f\) n’est pas nulle. Donc il existe (au moins) \(c \in [a, b]\) tel que \(f(c) \neq 0\) et puisque \(f\) est positive \(f(c) > 0\).

\sphinxAtStartPar
1\sphinxhyphen{} Si \(c \in ]a, b[\). Puisque la fonction est continue en \(c\) alors pour tout \(\epsilon >0\) il existe \(\alpha > 0\) tel que \(\forall x \in [a, b],~~ |x-c| \leq \alpha \Rightarrow |f(x) - f(x)|\leq \epsilon\).

\sphinxAtStartPar
On pose \(\epsilon = \dfrac{f(x)}{2}\)

\sphinxAtStartPar
alors il existe \(\alpha>0\) tel que pour tout \(x\in [a, b], ~~ x \in ]c-\alpha, c +\alpha[ \Rightarrow f(x)\leq f(c)-\epsilon > 0\).

\sphinxAtStartPar
On pose \(\beta = max(\dfrac{a+c}{2}, c-\alpha), \gamma = max(\dfrac{b+c}{2}, c+\alpha)\). On a
\begin{equation*}
\begin{split}
\begin{aligned}
\int_{[a, b]}f &= \int_{[a, \beta]}f + \int_{[\beta, \gamma]}f + \int_{[\gamma, b]}f \\ \\
&\geq  \int_{[\beta, \gamma]}f \\ \\
&\geq  (\gamma -\beta )\dfrac{f(c)}{2}> 0
\end{aligned}
\end{split}
\end{equation*}
\sphinxAtStartPar
2\sphinxhyphen{} Si \(c =a\) ou \(c=b\). La fonction \(f\) est continue en \(c\) et \(f(c)>0\). Donc elle est strictement positive au voisinage de \(c\). Ceci dit, il existe un réel \(d \in ]a, b[\) tel f(d)\textgreater{}0. On répète les mêmes étapes précédentes mais cette fois avec \(d\) et on aboutit a \(\int_{[a, b]}f >0\)

\sphinxAtStartPar
Ce qui est absurde.

\sphinxAtStartPar
Donc \(f\) est nulle.
\end{sphinxadmonition}

\begin{sphinxadmonition}{warning}{Avertissement:}
\sphinxAtStartPar
Les deux hypothèses (continuité et positivité) sont nécessaires pour que le résultat soit vrai.
\end{sphinxadmonition}

\begin{sphinxadmonition}{note}{Corollaire}

\sphinxAtStartPar
Si \(f\) et \(g\) sont deux fonctions continues sur \([a, b]\), alors:
\begin{equation*}
\begin{split}
\left (\int_{[a, b]}fg \right ) ^2 = \int_{[a, b]} f^2 \int_{[a, b]}g^2
\end{split}
\end{equation*}
\sphinxAtStartPar
Si et seulement si, \(f\) et \(g\) sont proportionnelles.
\end{sphinxadmonition}

\begin{sphinxadmonition}{note}{Démonstration}
\begin{itemize}
\item {} 
\sphinxAtStartPar
Si \(f\) et \(g\) sont proportionnelles, donc il existe \(\lambda \in \mathbb R\) tel que \(f=\lambda g\) où  \(g=\lambda f\). Supposons par exemple que \(f=\lambda g\).

\end{itemize}

\sphinxAtStartPar
Alors,
\begin{equation*}
\begin{split}
\left(\int_{[a, b]} fg \right)^2 = \lambda^2\left(\int_{[a, b]} g^2\right)^2 = \int_{[a, b]} f^2\int_{[a, b]} g^2
\end{split}
\end{equation*}
\sphinxAtStartPar
Supposons maintenant que
\begin{equation*}
\begin{split}
\left (\int_{[a, b]}fg \right ) ^2 = \int_{[a, b]} f^2 \int_{[a, b]}g^2
\end{split}
\end{equation*}
\sphinxAtStartPar
On pose
\begin{equation*}
\begin{split}
P(\lambda) = \int_{[a, b]}(f+\lambda g)^2 = \lambda^2\int_{[a, b]}g^2 + 2\lambda \int_{[a, b]}fg + \int_{[a, b]}f^2
\end{split}
\end{equation*}\begin{itemize}
\item {} 
\sphinxAtStartPar
\(si \int_{[a, b]}g^2 = 0\) alors \(g^2\) est nulle (puisqu’elle est continue est positive avec intégrale nulle) donc \(g\) est nulle. Donc \(g= 0\times f\). Par suite \(f\) et \(g\) sont proportionnelles.

\item {} 
\sphinxAtStartPar
sinon, le polynôme \(P\) a un discriminent nul. Donc il existe \(\lambda_0\) tel que \(P(\lambda_0)=0\).

\end{itemize}

\sphinxAtStartPar
Donc \(P(\lambda_0) = \int_{[a, b]}(f+\lambda_0 g)^2 = 0\). La fonction \((f+\lambda_0 g)^2\) est continue et positive avec intégrale nulle donc elle est nulle. Par suite \(f=\lambda_0 g\). Les deux fonctions sont proportionnelles.
\end{sphinxadmonition}


\subsection{Invariance par translation}
\label{\detokenize{pptint:invariance-par-translation}}
\begin{sphinxadmonition}{note}{Proposition}

\sphinxAtStartPar
Soient \(f\) une fonction continue par morceaux sur \([a, b]\) et \(\alpha\) un réel.
La fonction \(f_\alpha\) définie sur \([a+\alpha, b+\alpha]\) par \(f_\alpha(x)f(x-\alpha)\) est continue par morceaux sur \([a+\alpha, b+\alpha]\). De plus :
\begin{equation*}
\begin{split}
\int_{[a, b]}f= \int_{[a+\alpha, b+\alpha]} f_{\alpha}
\end{split}
\end{equation*}\end{sphinxadmonition}

\begin{sphinxadmonition}{note}{Démonstration}

\sphinxAtStartPar
On va dabord montrer le resultat pour une fonction en escalier pouis, pour une fonction continue par morceaux.
\begin{itemize}
\item {} 
\sphinxAtStartPar
Si \(f\) est une fonction en escalier sur \([a, b]\):

\end{itemize}

\sphinxAtStartPar
soit \(u=(x_i)_{i=0}^n\) une subdivision adaptee a \(f\). Alors on peut facilement montrer que la famille \((y_i)_{i=0}^n\) avec \(y_i = x_i +\alpha\) est une subdivision de \([a+\alpha, b+\alpha]\).

\sphinxAtStartPar
Si \(f\) prend la valeur \(c_i\) sur \(]x_{i-1}, x_i[\) alors \(f_\alpha\) vaut aussi \(c_i\) sur \(]x_{i-1}+\alpha,  x_i+\alpha[\). Donc \(f_\alpha\) est une fonction en escalier sur \([a+\alpha, b+\alpha]\).

\sphinxAtStartPar
De plus,
\begin{equation*}
\begin{split}
\int_{[a+\alpha, b+\alpha]} f_\alpha = \sum_{i=1}^n (y_i - y_{i-1})c_i = \sum_{i=1}^n (x_i - x_{i-1})c_i = \int_{[a, b]} f
\end{split}
\end{equation*}
\sphinxAtStartPar
Maintenant si \(f\) est continue par morceaux,

\sphinxAtStartPar
soit \(u=(x_i)_{i=0}^n\) une subdivision adaptee a \(f\). La famille \((y_i)_{i=0}^n\) avec \(y_i = x_i +\alpha\) est une subdivision de \([a+\alpha, b+\alpha]\).

\sphinxAtStartPar
\(f\) est continue sur \(]x_{i-1}, x_i[\) est admet des limites finies en \(x_{i-1}\) et \(x_{i}\). Donc, \(f_\alpha\) est continue sur \(]y_{i-1}, y_i[=]x_{i-1}+\alpha, x_i+\alpha[\) est admet des limites finies en \(y_{i-1}\) et \(y_{i}\). Donc  \(f_\alpha\) est continue par morceaux sur \([a+\alpha, b+\alpha]\).

\sphinxAtStartPar
si \(\varphi \in \mathcal E^-(f)\) donc \(\varphi \leq f\) donc \(\varphi_\alpha  \leq f_\alpha\)

\sphinxAtStartPar
donc \(\left\{\varphi_\alpha ~~ | ~~ \varphi \in \mathcal E^-(f) \right\} \subset \mathcal E^-(f_\alpha)\)

\sphinxAtStartPar
D’autre part, si \(\psi \in  \mathcal E^-(f_\alpha)\) donc il existe \(\phi \in  \mathcal E^-(f)\) telle que \(\psi = \varphi_\alpha\).

\sphinxAtStartPar
Donc, \(\mathcal E^-(f_\alpha)= \left\{\varphi_\alpha ~~ | ~~ \varphi \in \mathcal E^-(f) \right\}\)

\sphinxAtStartPar
Par suite
\begin{equation*}
\begin{split}
\begin{aligned}
\int_{[a+\alpha, b+\alpha]} f_{\alpha} &= sup \left\{\int_{[a+\alpha, b+\alpha]} \psi ~~| ~~ \psi \in \mathcal E^-(f_\alpha) \right\} \\ \\
&= sup \left\{\int_{[a+\alpha, b+\alpha]} \varphi_\alpha ~~| ~~ \varphi \in \mathcal E^-(f) \right\} \\ \\
&= sup \left\{\int_{[a, b]} \varphi ~~| ~~ \varphi \in \mathcal E^-(f) \right\} = \int_{[a, b]} f
\end{aligned}
\end{split}
\end{equation*}\end{sphinxadmonition}

\sphinxAtStartPar
Soient \(T>0\) et \(f\) une fonction \(T\)\sphinxhyphen{}périodique et continue par morceaux sur une période et donc sur tout segment de \(\mathbb R\). Nous avons :
\begin{equation*}
\begin{split}
\forall a \in \mathbb R, \int_{[a, a+ T]} f = \int_{[0, T]}f
\end{split}
\end{equation*}

\section{Somme de Riemann}
\label{\detokenize{sumreiman:somme-de-riemann}}\label{\detokenize{sumreiman::doc}}
\begin{sphinxadmonition}{note}{Définition}

\sphinxAtStartPar
Soient \(u=(x_i)_{i=0}^n\) une subdivision de \([a, b]\) et \(v=(y_i)_{i=1}^n\) une famille de points de \([a, b]\) telle que :
\begin{equation*}
\begin{split}
\forall i \in \{1, \ldots, n\}, y_i \in [x_{i-1}, x_i]
\end{split}
\end{equation*}
\sphinxAtStartPar
Si \(f\) est une fonction continue par morceaux sur \([a, b]\), on appelle \sphinxstylestrong{somme de Riemann} de \(f\) associée a la subdivision \(u\) et a la suite \(v\), la quantité:
\begin{equation*}
\begin{split}
\sigma(f, u, v)= \sum_{i=1}^n(x_i-x_{i-1})f(y_i)
\end{split}
\end{equation*}\end{sphinxadmonition}

\sphinxAtStartPar
Pour le cas d’une fonction en escalier (qui est une fonction continue par morceaux), la somme de Riemann \(\sigma(f, u, v)\) est égale a son intégrales.

\sphinxAtStartPar
Les cas les plus utilises sont ceux ou la subdivision \(u=(x_i)_{i=0}^n\) est pas constant, c’est\sphinxhyphen{}à\sphinxhyphen{}dire lorsque \(x_i = a+i\dfrac{b-a}{n}\)
\begin{itemize}
\item {} 
\sphinxAtStartPar
Considérons une subdivision régulière \(u=(x_i)_{i=0}^n\)  \(v=(y_i)_{i=1}^n\) une famille de points de \([a, b]\) telle que :

\end{itemize}
\begin{equation*}
\begin{split}
\begin{cases}
x_i = a + i\dfrac{b-a}{n},  ~~ 0\leq i \leq n \\
y_i = x_{i-1} , ~~ 1\leq i \leq n \\
\end{cases}
\end{split}
\end{equation*}
\sphinxAtStartPar
La somme de Riemann correspondante s’écrit :
\begin{equation*}
\begin{split}
\sigma(f, u, v) = \dfrac{b-a}{n}\sum_{i=0}^{n-1} f(a+i\dfrac{b-a}{n})
\end{split}
\end{equation*}
\noindent{\hspace*{\fill}\sphinxincludegraphics[width=400\sphinxpxdimen]{{sumreimaninf}.PNG}\hspace*{\fill}}
\begin{itemize}
\item {} 
\sphinxAtStartPar
Considérons une subdivision régulière \(u=(x_i)_{i=0}^n\)  \(v=(y_i)_{i=1}^n\) une famille de points de \([a, b]\) telle que :

\end{itemize}
\begin{equation*}
\begin{split}
\begin{cases}
x_i = a + i\dfrac{b-a}{n},  ~~ 0\leq i \leq n \\
y_i = x_{i} , ~~ 1\leq i \leq n \\
\end{cases}
\end{split}
\end{equation*}
\sphinxAtStartPar
La somme de Riemann correspondante s’écrit :
\begin{equation*}
\begin{split}
\sigma(f, u, v) = \dfrac{b-a}{n}\sum_{i=1}^n f(a+i\dfrac{b-a}{n})
\end{split}
\end{equation*}
\noindent{\hspace*{\fill}\sphinxincludegraphics[width=400\sphinxpxdimen]{{sumreimansup}.PNG}\hspace*{\fill}}

\begin{sphinxadmonition}{note}{Proposition (résultat admis)}

\sphinxAtStartPar
Soit \(f\) une application continue sur \([a, b]\),

\sphinxAtStartPar
Pour tout \(\epsilon > 0\), il existe \(\eta > 0\) tel que pour toute subdivision \(u=(x_i)_{i=0}^n\) et pour toute famille \(v=(y_i)_{i=1}^n\) qvec \(y_i \in [x_{i-1}, x_i]\) pour tout \(i \in \{1,\ldots,n\}\) on a:
\begin{equation*}
\begin{split}
\delta(u)\leq \eta \Rightarrow \left|\int_{[a, b]} f - \sigma(f, u, v)\right| \leq \epsilon
\end{split}
\end{equation*}\end{sphinxadmonition}

\begin{sphinxadmonition}{note}{Proposition}

\sphinxAtStartPar
Soit \(f\) une application continue sur \([a, b]\),

\sphinxAtStartPar
Pour tout \(\epsilon>0\), il existe \(\eta>0\) tel que pour toute subdivision \(u=(x_i)_{i=0}^n\) et pour toute famille \(v=(y_i)_{i=1}^n\) avec \(y_i \in [x_{i-1}, x_i]\) pour tout \(i \in \{1,\ldots,n\}\) on a:
\begin{equation*}
\begin{split}
\delta(u)\leq \eta \Rightarrow \left|\int_{[a, b]} f - \sigma(f, u, v)\right| \leq \epsilon
\end{split}
\end{equation*}\end{sphinxadmonition}

\sphinxAtStartPar
Les sommes de Riemann sont aussi proches que l’on veut de son intégrale quand le module de la subdivision tend vers 0.


\section{Fonctions continue par morceaux sur un intervalle}
\label{\detokenize{fcmint:fonctions-continue-par-morceaux-sur-un-intervalle}}\label{\detokenize{fcmint::doc}}
\sphinxAtStartPar
Dans cette partie, \(I\) désigne un intervalle de \(\mathbb R\).

\begin{sphinxadmonition}{note}{Définition}

\sphinxAtStartPar
Soit \(f\) une fonction définie sur \(I\). On dit que \(f\) est continue par morceaux sur \(I\) si elle est continue par morceaux sur tout segment de \(I\) (\([a, b]\) avec \(a, b \in I\) et \(a<b\)).
\end{sphinxadmonition}

\begin{sphinxadmonition}{note}{Exemples}

\sphinxAtStartPar
1\sphinxhyphen{} Une fonction continue sur \(I\) est continue par morceaux sur \(I\).
2\sphinxhyphen{} La fonction \( x \to x-E(x) \) est continue par morceaux sur \(\mathbb R\).
3\sphinxhyphen{} La fonction \(f\) définie sur \(\mathbb R\) par :
\begin{equation*}
\begin{split}f(0)=0 \mbox{  et  } \forall x \in [-1, 1] \setminus \{0\}, f(x)=\dfrac{1}{x}
\end{split}
\end{equation*}
\sphinxAtStartPar
n’est pas continue morceaux sur \(\mathbb R\) puisque elle n’est pas continue par morceaux sur \([-1, 1]\). Cependant, elle est continue par morceaux sur \(\mathbb R^*_+\) et \(\mathbb R^*_-\) car elle continue sur chacun de ces intervalles.
\end{sphinxadmonition}

\sphinxAtStartPar
\sphinxstylestrong{Notations}

\sphinxAtStartPar
Soient \(f\) une fonction continue par morceaux sur un intervalle \(I\), ainsi que \(a\) et \(b\) deux éléments de \(I\) ( a partir de maintenant, on a pas nécessairement \(a<b\)) On adopte les notations suivantes:
\begin{itemize}
\item {} 
\sphinxAtStartPar
si \(a<b\), \(\int_a^b f(x)dx = \int_{[a,b]} f\)

\item {} 
\sphinxAtStartPar
si \(a>b\), \(\int_a^b f(x)dx = -\int_{[b,a]} f\)

\item {} 
\sphinxAtStartPar
si \(a = b\), \(\int_a^b f(x)dx =0\)

\end{itemize}

\begin{sphinxadmonition}{warning}{Avertissement:}
\sphinxAtStartPar
Le résultat :
\begin{equation*}
\begin{split}
f \leq g \Rightarrow \int_a^b f(x)dx \leq \int_a^b g(x)dx
\end{split}
\end{equation*}
\sphinxAtStartPar
n’est pas valide que lorsque \(a\leq b\).
\end{sphinxadmonition}

\begin{sphinxadmonition}{note}{Proposition (Relation de Chasles)}

\sphinxAtStartPar
Si \(f\) est continue par morceaux sur un intervalle I, alors :
\begin{equation*}
\begin{split}
\forall a, b, c \in I, ~~ \int_a^b f(x)dx = \int_a^c f(x)dx + \int_c^b f(x)dx
\end{split}
\end{equation*}\end{sphinxadmonition}

\begin{sphinxadmonition}{note}{Démonstration}

\sphinxAtStartPar
Nous allons traiter le cas ou \(a=b=c\), \(a<c<b\) et \(a<b<c\). Les autres casd (\(b<a<c\), \(b<c<a\), \(c<a<b\) et \(c<b<a\)) sont similaire aux deux premiers cas.
\begin{itemize}
\item {} 
\sphinxAtStartPar
si \(a=b=c\) chaque intégrale vaut 0, le résultat est donx trivial.

\item {} 
\sphinxAtStartPar
si \(a<c<b\), c’est le cas qu’on a vu dans la proposition de la Relation de Chasles pour le cas d’une fonction continue par morceaux sur \([a, b]\).

\item {} 
\sphinxAtStartPar
si \(a<b<c\), on applique la même proposition (Relation de Chasles) pour \(f\) qui est continue par morceaux sur  \([a, c]\).

\end{itemize}

\sphinxAtStartPar
Donc
\begin{equation*}
\begin{split}
\int_a^c f(x)dx = \int_a^b f(x)dx + \int_b^c f(x)dx
\end{split}
\end{equation*}
\sphinxAtStartPar
Or \(\int_b^c f(x)dx = - \int_c^b f(x)dx\). Donc
\begin{equation*}
\begin{split}
\int_a^c f(x)dx = \int_a^b f(x)dx - \int_b^c f(x)dx
\end{split}
\end{equation*}
\sphinxAtStartPar
Et par suite
\begin{equation*}
\begin{split}
\int_a^b f(x)dx = \int_a^c f(x)dx + \int_c^b f(x)dx
\end{split}
\end{equation*}\end{sphinxadmonition}

\begin{sphinxadmonition}{note}{Proposition}

\sphinxAtStartPar
Si \(f\) est continue par morceaux et bornée sur \(I\), on :
\begin{equation*}
\begin{split}
\forall a, b \in I, ~~ \left|\int_a^b f(x)dx\right| \leq |b-a| sup_{I} |f|
\end{split}
\end{equation*}\end{sphinxadmonition}

\begin{sphinxadmonition}{note}{Démonstration}

\sphinxAtStartPar
Nous avons 3 cas : \(a=b\), \(a<b\) et \(a>b\).

\sphinxAtStartPar
1\sphinxhyphen{} si \(a=b\) alors \$\textbackslash{}left|\textbackslash{}int\_a\textasciicircum{}b f(x)dx\textbackslash{}right| = 0|, le résultat est immédiat.

\sphinxAtStartPar
2\sphinxhyphen{} si \(a<b\), la fonction \(f\) est continue par morceaux sur \([a, b]\). Donc, nous avons \(\left|\int_{[a, b]}f \right| \leq (b-a)sup_{[a, b]}|f|\)

\sphinxAtStartPar
Donc \(\left|\int_{a}^bf(x)dx \right| \leq (b-a)sup_{[a, b]}|f|\)

\sphinxAtStartPar
Or \(sup_{[a, b]}|f| \leq sup_{I}|f|\)

\sphinxAtStartPar
Donc, \(\left|\int_{a}^bf(x)dx \right| \leq (b-a)sup_{I}|f|\)

\sphinxAtStartPar
3\sphinxhyphen{} si \(a>b\), en applique les mêmes étapes précédentes pour la fonction \(f\) est continue par morceaux sur \([a, b]\) et on reçoit \(\left|\int_{b}^af(x)dx \right| \leq (b-a)sup_{I}|f|\)

\sphinxAtStartPar
Et puisque \(\int_a^b f(x)dx = - \int_b^a f(x)dx\) donc \(\left|\int_a^b f(x)dx\right| = \left|- \int_b^a f(x)dx\right| = \left| \int_b^a f(x)dx\right|\).

\sphinxAtStartPar
En fin, \(\left|\int_{a}^bf(x)dx \right| \leq (b-a)sup_{I}|f|\)
\end{sphinxadmonition}


\section{Exercices}
\label{\detokenize{exo1:exercices}}\label{\detokenize{exo1::doc}}

\subsection{Exercice 1}
\label{\detokenize{exo1:exercice-1}}
\sphinxAtStartPar
1\sphinxhyphen{} Soient \(m, n \in \mathbb Z\), et \(f\) une application telle que :
\begin{equation*}
\begin{split}
\begin{array}{ccccc}
f & : & [m, n] & \to & \mathbb R \\
 & & x & \mapsto & E(x)\\
\end{array}
\end{split}
\end{equation*}
\sphinxAtStartPar
Calculer
\begin{equation*}
\begin{split}
\int_{[m, n]} f
\end{split}
\end{equation*}

\subsection{Exercice 2}
\label{\detokenize{exo1:exercice-2}}
\sphinxAtStartPar
Soit \(f\) une application telle que :
\begin{equation*}
\begin{split}
\begin{array}{ccccc}
f & : & [-1, 2] & \to & \mathbb R \\
 & & x & \mapsto &  x|x|\\
\end{array}
\end{split}
\end{equation*}
\sphinxAtStartPar
Calculer
\begin{equation*}
\begin{split}
\int_{[-1, 2]} f dx
\end{split}
\end{equation*}

\subsection{Exercice 3}
\label{\detokenize{exo1:exercice-3}}
\sphinxAtStartPar
Soit \(a\in \mathbb R\):
Soit fonction \(f_a\) qui, pour tout \(x \in [0, 1]\), définie par \(f_a(x) = min(a, x)\).

\sphinxAtStartPar
Calculer
\begin{equation*}
\begin{split}\int_{[0, 1]} f_a
\end{split}
\end{equation*}

\subsection{Exercice 4}
\label{\detokenize{exo1:exercice-4}}
\sphinxAtStartPar
Soit \(f\) la fonction définie sur \([0, 4]\) par
\begin{equation*}
\begin{split}
f(x) =
\begin{cases}
-1& \mbox{ si } x=0 \\ \\ 
1 & \mbox{ si } 0<x<1 \\
3 & \mbox{ si } x=1 \\
-2&  \mbox{ si } 1< x \leq 2 \\
4 & \mbox{ si } 2< x \leq 4
\end{cases}
\end{split}
\end{equation*}\begin{enumerate}
\sphinxsetlistlabels{\arabic}{enumi}{enumii}{}{.}%
\item {} 
\sphinxAtStartPar
Vérifier que \(f\) est une fonction en escalier ?

\item {} 
\sphinxAtStartPar
Donner deux subdivisions adaptées à \(f\) (notees \(\sigma\) et \(\sigma^{'}\)).

\item {} 
\sphinxAtStartPar
Calculer \(I(\sigma, f)\) et \(I(\sigma^{'}, f)\).

\end{enumerate}


\subsection{Exercice 5}
\label{\detokenize{exo1:exercice-5}}
\sphinxAtStartPar
Soient les fonctions définies sur \(\mathbb R\),
\begin{equation*}
\begin{split}
f(x)=x \mbox{ et } g(x)= x^2
\end{split}
\end{equation*}
\sphinxAtStartPar
Est\sphinxhyphen{}ce que ces fonctions sont continue par morceaux sur tout intervalle ferme borne de \(\mathbb R\)?

\sphinxAtStartPar
En utilisant les sommes de Riemann, calculer les intégrales
\begin{equation*}
\begin{split}
\int_0^1 f(x)dx, \int_1^2 g(x)dx
\end{split}
\end{equation*}

\chapter{Intégration et Dérivation}
\label{\detokenize{integrationd:integration-et-derivation}}\label{\detokenize{integrationd::doc}}
\sphinxAtStartPar
Le présent Chapitre contiendra :
\begin{itemize}
\item {} 
\sphinxAtStartPar
Premitive et intégrale des fonctions continues

\item {} 
\sphinxAtStartPar
Methodes de calcul des premitives

\end{itemize}


\section{Primitive et intégrale des fonctions continues}
\label{\detokenize{pintfc:primitive-et-integrale-des-fonctions-continues}}\label{\detokenize{pintfc::doc}}
\sphinxAtStartPar
Dans tout ce qui suit, \(I\) désigne un intervalle de \(\mathbb R\) contenant au moins deux points distincts.


\subsection{Primitive d’une fonction continue sur un intervalle}
\label{\detokenize{pintfc:primitive-d-une-fonction-continue-sur-un-intervalle}}
\begin{sphinxadmonition}{note}{Définition}

\sphinxAtStartPar
Soit \(f\) est une fonction de \(I\) dans \(\mathbb R\) continue sur \(I\). On appelle primitive de \(f\) sur \(I\) toute fonction de \(I\) dans \(\mathbb R\), dérivable sur \(I\) et dont la dérivée est \(f\).
\end{sphinxadmonition}

\begin{sphinxadmonition}{note}{Proposition}

\sphinxAtStartPar
Soit \(f\) est une fonction de \(I\) dans \(\mathbb R\) continue sur \(I\).

\sphinxAtStartPar
Si \(F\) est une primitive de \(f\) sur \(I\), alors les primitives de \(f\) sur \(I\) sont les fonctions \(F+\lambda\) avec \(\lambda \in \mathbb R\).
\end{sphinxadmonition}

\begin{sphinxadmonition}{note}{Démonstration}

\sphinxAtStartPar
Les fonctions \(F+\lambda\) sont dérivables sur \(I\) est leurs dérivées valent \(f\). Donc \(F+\lambda\) sont des primitives de \(f\).

\sphinxAtStartPar
Soit \(G\) une primitive de \(f\) donc \(G-F\) a une dérivée nulle sur \(I\). Donc \(G-F\) est une constante sur \(I\).
\end{sphinxadmonition}


\subsection{Primitives usuelles}
\label{\detokenize{pintfc:primitives-usuelles}}
\sphinxAtStartPar
Les deux tableaux suivants contiennent les primitives des fonctions usuelles :

\noindent{\hspace*{\fill}\sphinxincludegraphics[width=500\sphinxpxdimen]{{prim1}.png}\hspace*{\fill}}

\noindent{\hspace*{\fill}\sphinxincludegraphics[width=500\sphinxpxdimen]{{prim2}.png}\hspace*{\fill}}

\begin{sphinxadmonition}{note}{Exemple}

\sphinxAtStartPar
Les primitives d’une fonction polynomiale de la forme
\begin{equation*}
\begin{split}
f(x) = \sum_{k=0}^n a_k x^k
\end{split}
\end{equation*}
\sphinxAtStartPar
sont de la forme \(F + \lambda\) avec:
\begin{equation*}
\begin{split}
F(x) = \sum_{k=0}^n \dfrac{a_k}{k+1} x^{k+1} ~~~ \mbox{ et } \lambda \in \mathbb{R}
\end{split}
\end{equation*}\end{sphinxadmonition}


\subsection{Théorème fondamental}
\label{\detokenize{pintfc:theoreme-fondamental}}
\begin{sphinxadmonition}{note}{Proposition}

\sphinxAtStartPar
Soient \(f\) une fonction continue par morceaux sur \(I\) et \(a\) un point de \(I\). La fonction \(F_a\) définie par :
\begin{equation*}
\begin{split}
F_a(x) = \int_a^x f(t) dt
\end{split}
\end{equation*}
\sphinxAtStartPar
est continue sur \(I\)
\end{sphinxadmonition}

\begin{sphinxadmonition}{note}{Théorème}

\sphinxAtStartPar
Soient \(F\) une fonction continue de \(I\) dans \(\mathbb{R}\) et \(a\) un point de \(I\). La fonction \(F_a\) définie par :
\begin{equation*}
\begin{split}
F_a(x) = \int_a^x f(t) dt
\end{split}
\end{equation*}
\sphinxAtStartPar
est une primitive de \(f\) sur \(I\). C’est l’unique primitive qui s’annule en \(a\).
\end{sphinxadmonition}

\begin{sphinxadmonition}{note}{Corollaire}

\sphinxAtStartPar
Soient f une fonction continue sur \(I\), ainsi que \(\alpha\) et \(\beta\) deux fonctions dérivables sur un intervalle \(J\) et a valeurs dans \(I\). La fonction définie sur \(J\) par :
\begin{equation*}
\begin{split}
\varphi(x) = \int_{\alpha(x)}^{\beta(x)} f(t) dt
\end{split}
\end{equation*}
\sphinxAtStartPar
est dérivable sur \(J\) et sa dérivée est:
\begin{equation*}
\begin{split}
\varphi^{'}(x) = \beta^{'}(x)f(\beta(x))- \alpha^{'}(x)f(\alpha(x))
\end{split}
\end{equation*}\end{sphinxadmonition}

\begin{sphinxadmonition}{note}{Exemple}
\begin{itemize}
\item {} 
\sphinxAtStartPar
Si \(f\) est une fonction continue par morceaux sur \(\mathbb{R}\) et périodique de période \(T\), alors:

\end{itemize}
\begin{equation*}
\begin{split}
g(x) = \int_x^{x+T}f(t)dt
\end{split}
\end{equation*}
\sphinxAtStartPar
est indépendante de \(x\) (constante), car:
\begin{equation*}
\begin{split}
g'(x) = f(x+T) - f(x) = 0
\end{split}
\end{equation*}\end{sphinxadmonition}

\begin{sphinxadmonition}{note}{Proposition}

\sphinxAtStartPar
Soient \(f\) une fonction continue par morceaux sur \(I\) et \(a\) un point de \(I\). Si \(F\) est une primitive de \(f\) sur \(I\), on a :
\begin{equation*}
\begin{split}
\int_a^b f(t) dt = F(b) - F(a)
\end{split}
\end{equation*}\end{sphinxadmonition}

\begin{sphinxadmonition}{note}{Exemple}
\begin{itemize}
\item {} 
\end{itemize}
\begin{equation*}
\begin{split}
\int_a^b e^{2x} dx = \dfrac{1}{2} (e^{2b} - e^{2a})
\end{split}
\end{equation*}\begin{itemize}
\item {} 
\end{itemize}
\begin{equation*}
\begin{split}
\int_0^\pi sin x dx = - cos \pi + cos 0 = 2
\end{split}
\end{equation*}\begin{itemize}
\item {} 
\sphinxAtStartPar
Soit \(f(x)=\alpha x + \beta\). Une primitive de \(f\) est \(x \mapsto \dfrac{\alpha}{2}x^2 + \beta x\), donx:

\end{itemize}
\begin{equation*}
\begin{split}
\int_a^bf(x)dx = \dfrac{\alpha}{2}(b^2 - a^2) + \beta (b-a) = (b-a)\dfrac{f(a)+ f(b)}{2}
\end{split}
\end{equation*}\end{sphinxadmonition}

\begin{sphinxadmonition}{note}{Corollaire}

\sphinxAtStartPar
Si \(f \in \mathcal C (I)\)(dérivable et sa dérivée est continue), alors pour \(a, x \in I\) on a:
\begin{equation*}
\begin{split}
f(x) - f(a) = \int_a^x f^{'}(t)dt
\end{split}
\end{equation*}\end{sphinxadmonition}

\sphinxAtStartPar
\sphinxstylestrong{Notations}:
\begin{itemize}
\item {} 
\sphinxAtStartPar
Dans ce qui suit, on va noter la différence de la fonction \(F\) entre \(a\) et \(b\): \([F(x)]_a^b\). Ceci dit,

\end{itemize}
\begin{equation*}
\begin{split}
\int_a^b f(x)dx = F(b) - F(a) =[F(x)]_a^b
\end{split}
\end{equation*}\begin{itemize}
\item {} 
\sphinxAtStartPar
Lorsque \(f\) est une fonction continue, la notation \(\int f(x)dx\) représente une primitive quelconque de la fonction \(f\) (\(\int f(x)dx = F(x) = Cst\)).

\end{itemize}


\section{Méthodes de calcul des primitives}
\label{\detokenize{methodcalp:methodes-de-calcul-des-primitives}}\label{\detokenize{methodcalp::doc}}

\subsection{Intégration par parties}
\label{\detokenize{methodcalp:integration-par-parties}}
\begin{sphinxadmonition}{note}{Proposition}

\sphinxAtStartPar
Si \(u\) et \(v\) sont deux fonctions de classe \(\mathcal C^1\) sur le segment \([a, b]\), on a:
\begin{equation*}
\begin{split}
\int_a^b u(t)v^{'}(t)dt = u(b)v(b)- u(a)v(a) - \int_a^b u^{'}(t)v(t)dt
\end{split}
\end{equation*}\end{sphinxadmonition}

\begin{sphinxadmonition}{note}{Démonstration}

\sphinxAtStartPar
Si \(u\) et \(v\) sont deux fonctions de classe \(\mathcal C^1\), alors \(uv\) est aussi de classe \(\mathcal C^1\).

\sphinxAtStartPar
Donc
\begin{equation*}
\begin{split}
u(b)v(b)- u(a)v(a) = \int_a^b (uv)^{'}(t)dt = \int_a^b u(t)v^{'}(t)dt + \int_a^b u^{'}(t)v(t)dt
\end{split}
\end{equation*}\end{sphinxadmonition}

\sphinxAtStartPar
\sphinxstylestrong{Remarque}: Dans un calcul de primitive, la formule d’intégration par parties s’écrit:
\begin{equation*}
\begin{split}
\int u(x)v^{'}(x)dx = u(x)v(x) - \int v(x)u^{'}(x)dx
\end{split}
\end{equation*}
\sphinxAtStartPar
La formule d’intégration par parties est en général utilisée pour:
\begin{itemize}
\item {} 
\sphinxAtStartPar
éliminer une fonction transcendantes dont la dérivée est plus simple comme par exemple les fonctions \(ln, arcsin, arctan,\ldots\)

\item {} 
\sphinxAtStartPar
calculer une intégrale par récurrence.

\end{itemize}

\begin{sphinxadmonition}{note}{Exemples}

\sphinxAtStartPar
1\sphinxhyphen{} sur \(\mathbb R_+^*\) on a :
\begin{equation*}
\begin{split}
\int ln x dx = \int (x)^{'}ln x dx = xln x - \int x\dfrac{1}{x} = xln x = x + Cst 
\end{split}
\end{equation*}
\sphinxAtStartPar
2\sphinxhyphen{} Sur \(\mathbb R\) on a:
\begin{equation*}
\begin{split}
\int arctan x dx = x arctan x - \int \dfrac{x}{1+x^2} dx = x arctan x - \dfrac{1}{2}ln (1+x^2) + Cst
\end{split}
\end{equation*}
\sphinxAtStartPar
3 \sphinxhyphen{} Pour calculer \(\int x^2 e^x dx\), on peut intégrer l’exponentielle et dériver le polynôme:
\begin{equation*}
\begin{split}
\int x^2 e^x dx = x^2e^x - 2\int xe^x dx
\end{split}
\end{equation*}
\sphinxAtStartPar
puis recommencer:
\begin{equation*}
\begin{split}
\int xe^x dx = xe^x - \int e^x dx
\end{split}
\end{equation*}
\sphinxAtStartPar
ce qui donne:
\begin{equation*}
\begin{split}
\int x^2 e^x dx = x^2e^x - 2(xe^x - \int e^x dx) = x^2e^x - 2xe^x + 2e^x + Cst
\end{split}
\end{equation*}\end{sphinxadmonition}


\subsection{Changement de variable}
\label{\detokenize{methodcalp:changement-de-variable}}
\begin{sphinxadmonition}{note}{Proposition}

\sphinxAtStartPar
Soient \(I\) et \(J\) deux intervalle de \(\mathbb R\), ainsi que \(f\) une fonction continue de \(I\) dans \(\mathbb R\) et \(\varphi\) une fonction de classe \(\mathcal C^1\) de \(J\) dans \(I\). Si \(\alpha\) et \(\beta\) sont deux éléments de \(J\), on a :
\begin{equation*}
\begin{split}
\int_{\varphi(\alpha)}^{\varphi(\beta)} f(t)dt = \int_\alpha^\beta f(\varphi(u)) \varphi^{'}(u)du
\end{split}
\end{equation*}\end{sphinxadmonition}

\begin{sphinxadmonition}{note}{Démonstration}

\sphinxAtStartPar
Comme \(f\) est continue sur \(I\), elle possède une primitive \(F\) et l’on a:
\begin{equation*}
\begin{split}
\int_{\varphi(\alpha)}^{\varphi(\beta)} f(t)dt = F(\varphi(\beta)) - F(\varphi(\alpha)) = F \circ\varphi(\beta) - F \circ\varphi(\alpha)
\end{split}
\end{equation*}
\sphinxAtStartPar
D’autre part, puisque les deux fonctions \(F\) et \(varphi\) sont de classe \(\mathcal C^1\) donc \(F \circ\varphi\) est aussi de classe \(\mathcal C^1\)

\sphinxAtStartPar
Donc on peut écrire : \( F \circ\varphi(\beta) - F \circ\varphi(\alpha) = F(\varphi(\beta)) - F(\varphi(\alpha)) = \int_\alpha^\beta  (F \circ\varphi)^{'}(u) du\).

\sphinxAtStartPar
Par suite :
\begin{equation*}
\begin{split}
\begin{aligned}
F(\varphi(\beta)) - F(\varphi(\alpha)) &= \int_\alpha^\beta  (F \circ\varphi)^{'}(u) du \\ \\
& = \int_\alpha^\beta  F^{'}(\varphi(u)) \varphi^{'}(u) du \\ \\
& = \int_\alpha^\beta f(\varphi(u)) \varphi^{'}(u)du
\end{aligned}
\end{split}
\end{equation*}\end{sphinxadmonition}

\sphinxAtStartPar
\sphinxstylestrong{Remarques}:
\begin{itemize}
\item {} 
\sphinxAtStartPar
la formule de changement de variable n’est que la formule de dérivation d’une fonction composée lue à l’envers.

\item {} 
\sphinxAtStartPar
Quand on utilise la formule de changement de variable avec les notations vues dans la proposition, on dit que l’on effectue le changement de variable \(t=\varphi(u)\) (d’où l’appellation changement de variable). On remplace alors \(t\) par \(\varphi(u)\) et \(dt\) par la différentielle \(\varphi^{'}(u)du\), ce qui rend le calcul assez naturel.

\item {} 
\sphinxAtStartPar
il faut faire attention lors de l’application de cette méthode, les bornes de l’intégral doivent être changées.

\end{itemize}

\begin{sphinxadmonition}{note}{Exemples}

\sphinxAtStartPar
1\sphinxhyphen{} Pour calculer l’intégrale :
\begin{equation*}
\begin{split}
\int_0^{\frac{\pi}{2}} sin^2 u cos u du
\end{split}
\end{equation*}
\sphinxAtStartPar
On pose \(t = sin u\), donc
\begin{equation*}
\begin{split}
\int_0^{\frac{\pi}{2}} sin^2 u cos u du = \int_0^{1} t^2 dt = \dfrac{1}{3}
\end{split}
\end{equation*}
\sphinxAtStartPar
2\sphinxhyphen{} Pour calculer l’intégrale
\begin{equation*}
\begin{split}
\int_{-1}^2 \sqrt{4-u^2} u du
\end{split}
\end{equation*}
\sphinxAtStartPar
On pose \(t = u^2\), donc :
\begin{equation*}
\begin{split}
\int_{-1}^2 \sqrt{4-u^2} u du = \dfrac{1}{2}\int_1^4 \sqrt{4-t}dt = [-\dfrac{1}{3}(4-t^2)^{\frac{3}{2}}]_1^4 = \sqrt{3}
\end{split}
\end{equation*}\end{sphinxadmonition}


\section{Exercices}
\label{\detokenize{exo2:exercices}}\label{\detokenize{exo2::doc}}

\subsection{Exercice 1}
\label{\detokenize{exo2:exercice-1}}
\sphinxAtStartPar
Trouver les primitives suivantes :
\begin{itemize}
\item {} 
\sphinxAtStartPar
a) \( \int (2x^2 + 3x - 5)dx\)

\item {} 
\sphinxAtStartPar
b) \(\int (x-1) dx\)

\item {} 
\sphinxAtStartPar
c) \(\int \dfrac{(1-x)^2}{\sqrt{x}}dx\)

\item {} 
\sphinxAtStartPar
d) \(\int \dfrac{x+3}{x+1}dx\)

\end{itemize}


\subsection{Exercice 2}
\label{\detokenize{exo2:exercice-2}}
\sphinxAtStartPar
Calculer :
\begin{itemize}
\item {} 
\sphinxAtStartPar
a) \(\int x\sqrt{1+x}dx\)

\item {} 
\sphinxAtStartPar
b) \(\int x^3e^{2x}dx\)

\item {} 
\sphinxAtStartPar
c) \(\int x^2ln(x)dx\)

\end{itemize}


\subsection{Exercice 3}
\label{\detokenize{exo2:exercice-3}}
\sphinxAtStartPar
Soit \(f\) une fonction continue de \([a, b]\) dans \(\mathbb R\) telle que:
\begin{equation*}
\begin{split}
\forall x \in [a, b], f(a+b-x)=f(x)
\end{split}
\end{equation*}
\sphinxAtStartPar
Montrer que :
\begin{equation*}
\begin{split}
\int_a^b xf(x)dx = \dfrac{a+b}{2}\int_a^b f(x)dx
\end{split}
\end{equation*}

\subsection{Exercice 4}
\label{\detokenize{exo2:exercice-4}}
\sphinxAtStartPar
En utilisant la reconnaissance de forme déterminer toutes les primitives des fonctions suivantes :
\begin{itemize}
\item {} 
\sphinxAtStartPar
\(f(x)=\dfrac{x}{1+x^2}\)

\item {} 
\sphinxAtStartPar
\( g(x) = \dfrac{e^{3x}}{1+e^{3x}}\)

\item {} 
\sphinxAtStartPar
\(h(x) = \dfrac{ln(x)}{x}\)

\item {} 
\sphinxAtStartPar
\(k(x) = cos(x)sin^2(x)\)

\item {} 
\sphinxAtStartPar
\(l(x) = \dfrac{1}{xln(x)}\)

\item {} 
\sphinxAtStartPar
\( m(x) = 3x\sqrt{1+x^2}\)

\end{itemize}


\subsection{Exercice 5}
\label{\detokenize{exo2:exercice-5}}\begin{itemize}
\item {} 
\sphinxAtStartPar
Calculer \(I_n = \int ln^n (x)dx\) pour \(n = 0; 1; 2\).

\item {} 
\sphinxAtStartPar
Calculer \(I_n\) en fonction de \(I_{n-1}\).

\end{itemize}


\subsection{Exercice 6}
\label{\detokenize{exo2:exercice-6}}
\sphinxAtStartPar
Calculer avec deux méthodes (reconnaissance de la forme et changement de variable) les primitives de la fonction suivantes :
\begin{itemize}
\item {} 
\sphinxAtStartPar
\(f(x) = cos^{1234}(x)sin(x)\)

\item {} 
\sphinxAtStartPar
\( g(x) = \dfrac{1}{xln(x)}\)

\end{itemize}


\subsection{Exercice 7}
\label{\detokenize{exo2:exercice-7}}
\sphinxAtStartPar
Calculer les intégrales suivantes :
\begin{itemize}
\item {} 
\sphinxAtStartPar
\(\int_0^{\frac{\pi}{2}} xsin(x)dx\) (par parties)

\item {} 
\sphinxAtStartPar
\(\int_0^1 \dfrac{e^x}{\sqrt{e^x}+1}\) (changement de variable)

\end{itemize}


\chapter{Intégrale impropre}
\label{\detokenize{integrationq:integrale-impropre}}\label{\detokenize{integrationq::doc}}
\sphinxAtStartPar
Le présent Chapitre contiendra :
\begin{itemize}
\item {} 
\sphinxAtStartPar
Définitions

\item {} 
\sphinxAtStartPar
Propriétés des intégrales convergentes

\item {} 
\sphinxAtStartPar
Cas de fonctions continues positives

\item {} 
\sphinxAtStartPar
Cas de fonctions de signe quelconque

\end{itemize}


\section{Intégration sur un intervalle quelconque}
\label{\detokenize{def:integration-sur-un-intervalle-quelconque}}\label{\detokenize{def::doc}}
\sphinxAtStartPar
Dans les chapitres précédents, nous avons définies \(\int_a^b f(t)dt\), pour une fonction continue ou continue par morceaux sur \([a, b]\). Dans le présent chapitre, étant donnée une fonction continue seulement sur \(]a, b[\), on cherche a donner un sens a \(\int_a^b f(t)dt\).


\subsection{Intégration sur un intervalle semi\sphinxhyphen{}ouvert}
\label{\detokenize{def:integration-sur-un-intervalle-semi-ouvert}}
\begin{sphinxadmonition}{note}{Définition}

\sphinxAtStartPar
Soit \(f\) une fonction continue sur l’intervalle \([a, b[\) ou \((-\infty <a < b \leq +\infty)\).
On dit que l’intégrale \(\int_a^b f(t)dt\) converge ou est convergent si la fonction \(x\mapsto \int_a^x f(t)dt\), qui est définie sur \([a, b[\)  possède une limite finie en \(b\).
\end{sphinxadmonition}

\sphinxAtStartPar
On note donc \(\int_a^b f(t)dt = \lim_{x\to b} \int_a^x f(t)dt\). Si cette intégrale n’est pas convergente, on dit qu’elle est convergente. L’intégrale \(\int_a^b f(t)dt\) est dite impropre (ou généralisée). On dit aussi qu’il y a une impropreté en \(b\). Ou encore que l’intégrale est généralisée en \(b\).

\sphinxAtStartPar
\sphinxstylestrong{Remarques :}
\begin{itemize}
\item {} 
\sphinxAtStartPar
Si f est continue sur \([a, b]\), on retrouve la définition de l’intégrale d’une fonction continue sur le segment \([a, b]\).

\item {} 
\sphinxAtStartPar
Si \(b\) est fini et \(f\) admet un prolongement par continuité en \(b\), l’intégrale \(\int_a^b f(t)dt\) converge, car si \(\tilde{f}\) est le prolongement par continuité de \(f\), on obtient:

\end{itemize}
\begin{equation*}
\begin{split}
\lim_{x \to b} \int_a^x f(t)dt = \lim_{x\to b} \int_a^x \tilde{f}(t)dt = \int_a^b \tilde{f}(t)dt
\end{split}
\end{equation*}
\sphinxAtStartPar
On dit qu’on a une « fausse impropreté » en \(b\).
\begin{itemize}
\item {} 
\sphinxAtStartPar
Si \(b=+\infty\), l’intégrale \(\int_a^b f(t)dt\) est toujours impropre.

\end{itemize}

\begin{sphinxadmonition}{note}{Proposition}

\sphinxAtStartPar
Si \(f\) est une fonction continue sur l’intervalle \([a, b[\), \((-\infty <a < b \leq +\infty)\), l’intégrale \(\int_a^b f(t)dt\) est convergent si, et seulement si, toute primitive \(F\) de \(f\) possède une limite finie en \(b\) et l’in a alors:
\begin{equation*}
\begin{split}
\int_a^b f(t)dt = \lim_{x\to b} F(x) - F(a)
\end{split}
\end{equation*}\end{sphinxadmonition}

\begin{sphinxadmonition}{note}{Preuve}

\sphinxAtStartPar
Nous avons \(\int_a^x f(t)dt=F(x)-F(a)\).
\end{sphinxadmonition}

\sphinxAtStartPar
\sphinxstylestrong{Remarques :}
\begin{itemize}
\item {} 
\sphinxAtStartPar
Dans le cas où on connait une primitive de \(f\), on peut ainsi déterminer la nature convergente ou divergente de l’intégrale et la calculer.

\item {} 
\sphinxAtStartPar
Il résulte de cette proposition que si \(c\in ]a, b[\), l’intégrale \(\int_a^b f(t)dt\) converge si, et seulement si, \(\int_c^b f(t)dt\) converge. La nature de l’intégrale \(\int_a^b f(t)dt\) ne dépend pas donc que du comportement de \(f\) au voisinage de l’impropre \(b\).

\end{itemize}

\begin{sphinxadmonition}{note}{Exemples}
\begin{itemize}
\item {} 
\sphinxAtStartPar
la fonction \( t \mapsto e^{-t}\) est continue sur \([0, +\infty[\) et a pour primitive la fonction \(t\mapsto -e^{-t}\). Comme \(\lim_{t \to +\infty} -e^{-t} = 0\), l’integrale \(\int_0^{+\infty} e^{-t}dt\) converge et

\end{itemize}
\begin{equation*}
\begin{split}
\int_0^{+\infty}e^{-t}dt = \lim_{t\to +\infty} -e^{-t} + e^{0} = 1
\end{split}
\end{equation*}
\sphinxAtStartPar
2\sphinxhyphen{} L’intégrale \(\int_0^{+\infty} sin(t)dt\) diverge, car la fonction \(-cos\), primitive de \(sin\) n’a pas de limite en \(+\infty\).
\end{sphinxadmonition}


\subsection{Reste d’une intégrale convergente}
\label{\detokenize{def:reste-d-une-integrale-convergente}}
\begin{sphinxadmonition}{note}{Définition}

\sphinxAtStartPar
Soit \(f\) une fonction continue sur l’intervalle \([a, b[\), \((-\infty <a < b \leq +\infty)\)

\sphinxAtStartPar
On suppose que l’intégrale \(\int_a^b f(t)dt\)  est convergente. On appelle reste de cette intégrale impropre l’intégrale \(\int_x^b f(t)dt\) ou \(x\in]a, b[\).
\end{sphinxadmonition}

\begin{sphinxadmonition}{note}{Proposition}

\sphinxAtStartPar
Soit \(f\) une fonction continue sur l’intervalle \([a, b[\), \((-\infty <a < b \leq +\infty)\) telle que l’intégrale \(\int_a^b f(t)dt\) converge, Le reste \(\int_x^b f(t)dt\) ou \(x\in ]a, b[\) a pour limite 0 quand \(x\) tend vers \(b\).
\end{sphinxadmonition}

\begin{sphinxadmonition}{note}{Preuve}

\sphinxAtStartPar
Nous avons \(\lim_{x\to b} \int_x^b f(t)dt = \lim_{x\to b} ( \lim_{u\to b} F(u)- F(x)) = \lim_{u\to b} F(u)- \lim_{x\to b}F(x) =0\)
\end{sphinxadmonition}

\sphinxAtStartPar
On définit de la même façon l’intégrale d’une fonction continue sur \(]a, b]\).

\begin{sphinxadmonition}{note}{Définition}

\sphinxAtStartPar
Soit \(f\) une fonction continue sur l’intervalle \(]a, b]\) ou \((-\infty \leq a < b < +\infty)\).
On dit que l’intégrale \(\int_a^b f(t)dt\) converge ou est convergent si la fonction \(x\mapsto \int_x^b f(t)dt\), qui est définie sur \(]a, b]\)  possède une limite finie en \(a\).
\end{sphinxadmonition}

\sphinxAtStartPar
On note donc \(\int_a^b f(t)dt = \lim_{x\to a} \int_x^b f(t)dt\).

\begin{sphinxadmonition}{note}{Proposition}

\sphinxAtStartPar
Soit \(f\) une fonction continue sur l’intervalle \(]a, b]\) ou \((-\infty \leq a < b < +\infty)\), l’intégrale \(\int_a^b f(t)dt\) est convergent si, et seulement si, toute primitive \(F\) de \(f\) possède une limite finie en \(a\) et l’on a alors:
\begin{equation*}
\begin{split}
\int_a^b f(t)dt = F(b) -\lim_{x\to a}F(x)
\end{split}
\end{equation*}\end{sphinxadmonition}

\begin{sphinxadmonition}{note}{Exemple}

\sphinxAtStartPar
Montrons la convergence de \(\int_0^1 ln(t)dt\), qui est impropre car la fonction \(ln\) est continue sur \(]0, 1]\) mais n’est pas définie en 0.

\sphinxAtStartPar
On obtient pour \(x\in ]0, 1]\), en intégrant par parties,

\sphinxAtStartPar
\(\int_x^1 ln(t)dt=[tln(t)]_{x}^1 - \int_x^1dt = xln(x)-1 + x\).

\sphinxAtStartPar
Comme \(\lim_{x\to 0} xln(x)=0\), on en déduit que \(\int_0^1ln(t)dt \) converge et \(\int_0^1ln(t)dt =-1\).
\end{sphinxadmonition}

\sphinxAtStartPar
\sphinxstylestrong{Remarques :} Comme précédemment, si \(f\) est continue sur \(]a, b]\), le reste de l’intégrale convergente \(\int_a^b f(t)dt\), qui est \(\int_a^x f(t)dt\), tend vers 0 quand \(x\) tend vers \(a\).


\subsection{Intégrale d’une fonction sur un intervalle ouvert}
\label{\detokenize{def:integrale-d-une-fonction-sur-un-intervalle-ouvert}}
\begin{sphinxadmonition}{note}{Définition}

\sphinxAtStartPar
Soit \(f\) continue sur \(]a, b[\) ou \((-\infty \leq a < b \leq +\infty)\), \(c \in ]a, b[\). On dit que l’intégrale \(\int_a^b f(t)dt\) converge si \(\int_a^c f(t)dt\) et \(\int_c^b f(t)dt\) convergent et l’on pose, en cas de convergence,
\begin{equation*}
\begin{split}
\int_a^b f(t)dt = \int_a^c f(t)dt + \int_c^b f(t)dt
\end{split}
\end{equation*}\end{sphinxadmonition}

\sphinxAtStartPar
Cette définition ne dépend pas du choix du point \(c\) comme le montre la proposition suivante:

\begin{sphinxadmonition}{note}{Proposition}

\sphinxAtStartPar
Soit \(f\) continue sur \(]a, b[\) ou \((-\infty \leq a < b \leq +\infty)\), \(c \in ]a, b[\). L’intégrale \(\int_a^b f(t)dt\) converge si, et seulement si, toute primitive \(F\) de \(f\) sur \(]a, b[\) possède une limite finie en \(a\) et \(b\) et en cas de convergence, on a
\begin{equation*}
\begin{split}
\int_a^b f(t)dt = \lim_b F - \lim_a F
\end{split}
\end{equation*}\end{sphinxadmonition}

\begin{sphinxadmonition}{note}{Théorème}

\sphinxAtStartPar
Soit \(\alpha\) un réel quelconque.

\sphinxAtStartPar
Si \(a\) est un réel positif, l’intégrale \(\int_0^a \dfrac{1}{t^\alpha} dt \) converge si, et seulement si, \(\alpha <1\); l’intégrale \(\int_a^{+\infty} \dfrac{1}{t^\alpha} dt \) converge si, et seulement si, \(\alpha > 1\).

\sphinxAtStartPar
Plus généralement, si \(a\) et \(b\) sont des réels tels \(a<b\), \(\int_a^b \dfrac{1}{(t-a)^{\alpha}}dt \) converge si, et seulement si \(\alpha <1\).
\end{sphinxadmonition}


\subsection{Propriétés des intégrales convergentes}
\label{\detokenize{def:proprietes-des-integrales-convergentes}}
\begin{sphinxadmonition}{note}{Définition}

\sphinxAtStartPar
Soit \(f : [a, b] \to \mathbb R\).

\sphinxAtStartPar
On suppose qu’il existe une subdivision \(a_0=a < a_1 < \ldots < a_p=b\) de \([a, b]\) telle que \(f\) soit définie et continue sur chaque intervalle \(]a_k, a_{k+1}[\)(\(0\leq k \leq (p-1)\)). On dit que l’intégrale \(\int_a^b f(t)dt\) converge si, pour \(0\leq k \leq p-1\), \(\int_{a_{k}}^{a_{k+1}} f(t)dt\) converge. Le cas échéant, on pose :
\begin{equation*}
\begin{split}
\int_a^b f(t)dt = \sum_{k=0}^{p-1} \int_{a_{k}}^{a_{k+1}} f(t)dt
\end{split}
\end{equation*}\end{sphinxadmonition}

\sphinxAtStartPar
\sphinxstylestrong{Remarques :}
\begin{itemize}
\item {} 
\sphinxAtStartPar
Dans la définition précédente, le résultat ne dépend pas de la subdivision choisie.

\item {} 
\sphinxAtStartPar
Si \(f\) est continue par morceaux alors les intégrales \(\int_{a_{k}}^{a_{k+1}} f(t)dt\) convergent car sa restriction sur \([a_k, a_{k+1}]\) possède un prolongement par continuité en \(a_k\) et \(a_{k+1}\).

\end{itemize}

\sphinxAtStartPar
Les théorèmes qui suivent sont des versions « intégrales impropres » des propriétés des intégrales sur un segment.

\begin{sphinxadmonition}{note}{Théorème}

\sphinxAtStartPar
Soit \(f\) et \(g\) deux fonctions définies et continue sur l’intervalle \([a, b]\), \((-\infty \leq a < b \leq +\infty)\) sauf en un nombre fini de points, \(\lambda\) et \(\mu\) deux réels. Si les intégrales \(\int_a^b f(t)dt\) et \(\int_a^b g(t)dt\) convergent, alors \(\int_a^b (\lambda f + \mu g)(t)dt\) converge et
\begin{equation*}
\begin{split}
\int_a^b (\lambda f + \mu g)(t)dt = \lambda\int_a^b f(t)dt + \mu\int_a^b g(t)dt
\end{split}
\end{equation*}\end{sphinxadmonition}

\begin{sphinxadmonition}{note}{Proposition}

\sphinxAtStartPar
Soit \(f\) une fonction définie et continue sur l’intervalle \([a, b]\), \((-\infty \leq a < b \leq +\infty)\) sauf en un nombre fini de points, \(c \in ]a, b[\). Si l’intégrale \(\int_a^b f(t)dt\) converge, alors \(\int_a^c f(t)dt\)  et \(\int_c^b f(t)dt\) convergent et
\begin{equation*}
\begin{split}
\int_a^b f(t)dt =\int_a^c f(t)dt + \int_c^b f(t)dt
\end{split}
\end{equation*}\end{sphinxadmonition}


\subsubsection{Positivité}
\label{\detokenize{def:positivite}}
\begin{sphinxadmonition}{note}{Proposition}

\sphinxAtStartPar
Soit \(f\) une fonction definie sur \(]a, b[\) (\(-\infty \leq a < b \leq +\infty\)).

\sphinxAtStartPar
Si \(\int_a^b f(t)dt\) converge et si \(f\) est positive sur \(]a, b[\), alors on a \(\int_a^b f(t)dt\geq 0\). Si de plus \(f\) n’est pas la fonction nulle, on obtient \(\int_a^b f(t)dt>0\).
\end{sphinxadmonition}

\begin{sphinxadmonition}{note}{Corollaire}

\sphinxAtStartPar
Soient \(f\) et \(g\) deux fonctions définies et continues sur \(]a, b[\) (\(-\infty \leq a < b \leq +\infty\)).
Si les intégrales \(\int_a^b f(t)dt\) et \(\int_a^b g(t)dt\) convergent et si \(f\leq g\), alors on a
\begin{equation*}
\begin{split}
\int_a^bf(t)dt \leq \int_a^b g(t)dt
\end{split}
\end{equation*}
\sphinxAtStartPar
Si de plus \(f\) n’est pas égale à \(g\). On obtient
\begin{equation*}
\begin{split}
\int_a^b f(t)dt < \int_a^b g(t)dt
\end{split}
\end{equation*}\end{sphinxadmonition}


\subsubsection{Intégration par parties}
\label{\detokenize{def:integration-par-parties}}
\begin{sphinxadmonition}{note}{Théorème}

\sphinxAtStartPar
Soient \(u\) et \(v\) deux fonctions de classe \(\mathcal C^1\) de \(]a, b[\) dans \(\mathbb R\) avec (\(-\infty \leq a < b \leq +\infty\)). Si le produit \(uv\) possède une limite finie en \(a\) et \(b\), les intégrales \(\int_a^b u^{'}(t)v(t)dt\) et \(\int_a^b u(t)v^{'}(t)dt\) ont même nature et, en cas de convergence, on dispose de l’égalité
\begin{equation*}
\begin{split}
\int_a^b u^{'}(t)v(t)dt= \lim_{b} uv - \lim_{a} uv - \int_a^b u(t)v^{'}(t)dt
\end{split}
\end{equation*}\end{sphinxadmonition}


\subsubsection{Changement de variables}
\label{\detokenize{def:changement-de-variables}}
\begin{sphinxadmonition}{note}{Théorème}

\sphinxAtStartPar
Soient \(\varphi\) une fonction de classe \(\mathcal C^1\), strictement monotone, réalisant une bijection de \(]\alpha, \beta[\) sur \(]a, b[\) et \(f : ]a, b[ \to \mathbb R\), continue. Les intégrales \(\int_\alpha^\beta f(\varphi(t))\varphi^{'}(t)dt\) et \(\int_a^b f(t)dt\) ont même nature et en cas de convergence
\begin{equation*}
\begin{split}
\int_a^b f(t)dt = \int_\alpha^\beta f(\varphi(t))\varphi^{'}(t)dt \mbox{ si } \varphi \mbox{ est croissante}
\end{split}
\end{equation*}\begin{equation*}
\begin{split}
\int_a^b f(t)dt = - \int_\alpha^\beta f(\varphi(t))\varphi^{'}(t)dt \mbox{ si } \varphi \mbox{ est décroissante}
\end{split}
\end{equation*}\end{sphinxadmonition}


\subsection{Cas des fonctions positives}
\label{\detokenize{def:cas-des-fonctions-positives}}
\sphinxAtStartPar
Pour la plupart des fonctions, on ne sait pas expliciter de primitive. C’est pourquoi il est utile de disposer de méthodes permettant de montrer la convergence des intégrales impropres sans en calculer la valeur. Dans cette section, on traite du cas où la fonction garde un signe constant au voisinage de l’impropreté.


\subsubsection{Condition nécessaire et suffisante de convergence}
\label{\detokenize{def:condition-necessaire-et-suffisante-de-convergence}}
\begin{sphinxadmonition}{note}{Théorème}

\sphinxAtStartPar
Soit \(f\) une fonction continue et positive sur \([a, b[\). L’intégrale \(\int_a^b f(t)dt\) converge si, et seulement si, la fonction \(x\mapsto \int_a^x f(t)dt\) est majorée sur \([a, b[\).

\sphinxAtStartPar
De même,

\sphinxAtStartPar
Si \(f\) une fonction continue et positive sur \(]a, b]\). L’intégrale \(\int_a^b f(t)dt\) converge si, et seulement si, la fonction \(x\mapsto \int_x^b f(t)dt\) est majorée sur \(]a, b]\).
\end{sphinxadmonition}


\subsubsection{Critère de comparaison}
\label{\detokenize{def:critere-de-comparaison}}
\begin{sphinxadmonition}{note}{Théorème}

\sphinxAtStartPar
Soient \(f\) et \(g\) deux fonctions continues sur \([a, b[\). On suppose qu’il existe \(c\in [a, b[\) tel que \(0\leq f(t)\leq g(t)\) pour tout \(t\in [a, b[\).

\sphinxAtStartPar
Si l’intégrale \(\int_a^b g(t)dt\) converge, alors \(\int_a^b f(t)dt\) converge.

\sphinxAtStartPar
Si l’intégrale \(\int_a^b f(t)dt\) diverge, alors \(\int_a^b g(t)dt\) diverge.
\end{sphinxadmonition}


\subsection{Cas des fonctions de signe quelconque}
\label{\detokenize{def:cas-des-fonctions-de-signe-quelconque}}

\subsubsection{Convergence absolue}
\label{\detokenize{def:convergence-absolue}}
\begin{sphinxadmonition}{note}{Définition}

\sphinxAtStartPar
Soit \(f : [a, b] \to \mathbb R\) une fonction continue (\(-\infty \leq a < b \leq +\infty\)). On dit que l’intégrale \(\int_a^b f(t)dt\) est absolument convergente si \(\int_a^b |f(t)|dt\) est convergente.
\end{sphinxadmonition}

\begin{sphinxadmonition}{note}{Théorème}

\sphinxAtStartPar
Soit \(f : [a, b] \to \mathbb R\) une fonction continue (\(-\infty \leq a < b \leq +\infty\)). si \(\int_a^b f(t)dt\) est absolument convergente, elle est convergente. De plus, on dispose de l’inégalité
\begin{equation*}
\begin{split}
\left | \int_a^b f(t)dt\right | \leq \int_a^b |f(t)|dt
\end{split}
\end{equation*}\end{sphinxadmonition}

\begin{sphinxadmonition}{note}{Exemple}

\sphinxAtStartPar
Pour \(\alpha >1\), l’intégrale \(\int_1^{+\infty} \dfrac{sin(t)}{t^\alpha}dt\) est absolument convergente car, pour \(t\geq 1\), \(\left |\dfrac{sin(t)}{t^\alpha} \right | \leq \dfrac{1}{t^\alpha}\) et \(\int_1^{+\infty} \dfrac{1}{t^\alpha}dt\) converge.
\end{sphinxadmonition}


\section{Exercices}
\label{\detokenize{exo3:exercices}}\label{\detokenize{exo3::doc}}

\subsection{Exercice 1}
\label{\detokenize{exo3:exercice-1}}
\sphinxAtStartPar
Étudier l’existence des intégrales suivantes :
\begin{itemize}
\item {} 
\sphinxAtStartPar
\(\int_0^1 ln(t)dt\)

\item {} 
\sphinxAtStartPar
\(\int_0^{+\infty} \dfrac{e^{-\sqrt{x}}}{\sqrt{x}} dx\)

\end{itemize}


\subsection{Exercice 2}
\label{\detokenize{exo3:exercice-2}}
\sphinxAtStartPar
Étudier l’existence des intégrales suivantes :
\begin{itemize}
\item {} 
\sphinxAtStartPar
\(\int_0^1 \dfrac{1}{t^2}dt\)

\item {} 
\sphinxAtStartPar
\(\int_0^2 \dfrac{1}{\sqrt{t}}dt\)

\item {} 
\sphinxAtStartPar
\(\int_0^3 \dfrac{1}{t^\frac{3}{2}}dt\)

\end{itemize}


\subsection{Exercice 3}
\label{\detokenize{exo3:exercice-3}}
\sphinxAtStartPar
Étudier l’existence des intégrales suivantes :
\begin{itemize}
\item {} 
\sphinxAtStartPar
\(\int_1^{+\infty} \dfrac{1}{t^2}dt\)

\item {} 
\sphinxAtStartPar
\(\int_2^{+\infty} \dfrac{1}{\sqrt{t}}dt\)

\item {} 
\sphinxAtStartPar
\(\int_2^{+\infty} \dfrac{1}{t^\frac{3}{2}}dt\)

\end{itemize}


\subsection{Exercice 4}
\label{\detokenize{exo3:exercice-4}}
\sphinxAtStartPar
Étudier l’existence des intégrales suivantes :
\begin{itemize}
\item {} 
\sphinxAtStartPar
\(\int_4^9 \dfrac{1}{(t-4)^2}dt\)

\item {} 
\sphinxAtStartPar
\(\int_4^9 \dfrac{1}{\sqrt{t-4}}dt\)

\end{itemize}


\subsection{Exercice 5}
\label{\detokenize{exo3:exercice-5}}
\sphinxAtStartPar
Étudier l’existence des intégrales suivantes :
\begin{itemize}
\item {} 
\sphinxAtStartPar
\(\int_0^{+\infty} e^{-t^2}dt\)

\item {} 
\sphinxAtStartPar
\(\int_0^{+\infty} (\dfrac{ln(x)}{x+e^x})dx\)

\item {} 
\sphinxAtStartPar
\(\int_1^{+\infty} (e^{-\sqrt{x^2-x}})dx\)

\item {} 
\sphinxAtStartPar
\(\int_0^{+\infty} (x^{-ln(x )})dx\)

\item {} 
\sphinxAtStartPar
\(\int_0^{1} \dfrac{x-1}{ln(x)}dx\)

\end{itemize}


\subsection{Exercice 6}
\label{\detokenize{exo3:exercice-6}}
\sphinxAtStartPar
Montrer que les intégrales généralisées \(\int_2^{+\infty}\dfrac{dx}{x+1}\) et \(\int_2^{+\infty}\dfrac{dx}{x-1}\) sont divergentes. Que peut\sphinxhyphen{}on dire de l’intégrale généralisée
\(\int_2^{+\infty}(\dfrac{1}{x+1}+\dfrac{1}{1-x})dx\)?


\subsection{Exercice 7}
\label{\detokenize{exo3:exercice-7}}
\sphinxAtStartPar
Déterminer la nature des intégrales suivantes.
\begin{itemize}
\item {} 
\sphinxAtStartPar
\(\int_1^{+\infty} \dfrac{1-cos(x)}{x^2}dx\)

\item {} 
\sphinxAtStartPar
\(\int_0^{+\infty} \dfrac{x^2}{x^{\frac{17}{2}}}dx\)

\item {} 
\sphinxAtStartPar
\(\int_0^{1} \dfrac{e^x}{x}dx\)

\item {} 
\sphinxAtStartPar
\(\int_0^{1} \dfrac{x^2+1}{x}dx\)

\end{itemize}


\chapter{Les Séries Numériques}
\label{\detokenize{Seriesc:les-series-numeriques}}\label{\detokenize{Seriesc::doc}}
\sphinxAtStartPar
Le présent Chapitre contiendra :
\begin{itemize}
\item {} 
\sphinxAtStartPar
Définitions

\item {} 
\sphinxAtStartPar
Séries à termes positifs

\item {} 
\sphinxAtStartPar
Séries alternées

\end{itemize}


\section{Les séries numériques}
\label{\detokenize{series:les-series-numeriques}}\label{\detokenize{series::doc}}

\subsection{Définitions et premières propriétés}
\label{\detokenize{series:definitions-et-premieres-proprietes}}
\begin{sphinxadmonition}{note}{Définition}

\sphinxAtStartPar
Soit \((u_n)_{n\in\mathbb N}\) une suite d’éléments de \(\mathbb R\) (ou \(\mathbb C\)). Pour tout \(n\in\mathbb N\) on pose \(S_n= \sum_{\substack{k=0}}^n u_k\) (\(n^{-ème}\) somme partielle). La série de terme général \(u_n\) est la suite \((S_n)_n\). \(S_n\) est la somme partielle de rang \(n\) de la série de terme général \(u_n\).
\begin{equation*}
\begin{split}
S_n=u_0+u_1+\dots+u_n
\end{split}
\end{equation*}\end{sphinxadmonition}

\sphinxAtStartPar
La série de terme général est notée \(\sum_{n\geq 0} u_n\) ou \(\sum u_n\).

\sphinxAtStartPar
On s’intéresse à la limite de \(\sum_{n\geq 0} u_n\) lorsque \(n\to +\infty\).

\sphinxAtStartPar
On dira que la série \(\sum u_n\) est :
\begin{itemize}
\item {} 
\sphinxAtStartPar
convergente (CV) si \(\lim_{n\to +\infty} S_n\) existe, et on note alors \(\sum_{n\geq 0}u_n\) cette limite ou encore \(\sum_{\substack{n=0}}^{+\infty}u_n\).

\item {} 
\sphinxAtStartPar
divergente (DIV) sinon.

\item {} 
\sphinxAtStartPar
absolument convergente (AC) si la série de terme général \(|u_n|\) (ou encore \(\sum_{n\geq 0} |u_n|\)) est convergente.

\end{itemize}

\sphinxAtStartPar
On dira aussi que la série converge simplement (CS) si elle converge mais pas absolument.

\sphinxAtStartPar
On peut définir de même la notion de convergence de la série \(\sum_{n\geq p} u_n\) si \(u_n\) n’est définie qu’a partir du rang \(p\):
\begin{equation*}
\begin{split}
\sum_{\substack{n\geq p}} u_n = u_p + u_{p+1}+ \ldots 
\end{split}
\end{equation*}
\sphinxAtStartPar
La modification d’un nombre fini de termes de la série ne change pas sa nature (CV, AC, DIV, CS).

\begin{sphinxadmonition}{note}{Proposition (Condition nécessaire)}

\sphinxAtStartPar
Si la série \(\sum_{n\geq 0}u_n\) converge alors \((u_n)\) converge vers 0.
\end{sphinxadmonition}

\sphinxAtStartPar
Toute série dont le terme général ne converge pas vers 0 est donc divergente. Par exemple, la série de terme général \(n\) (\(\sum_{n\geq 0}n\)) est divergente. Par contre si le terme général converge vers 0 cela ne veut pas dire que la série est convergente. La convergence de la suite de terme général vers 0 est condition nécessaire mais pas suffisante.


\subsection{Série harmonique \protect\(\sum_{n\geq 1} \dfrac{1}{n}\protect\)}
\label{\detokenize{series:serie-harmonique-sum-n-geq-1-dfrac-1-n}}
\sphinxAtStartPar
La série harmonique (de terme général \(\dfrac{1}{n}\)) est divergente. En effet, pour montrer ca, on va montrer que la suite des somme partielle est plus grandes qu’une suite qui est divergente.

\sphinxAtStartPar
Soit \(k\geq 1\).

\sphinxAtStartPar
La fonction \(f: x\mapsto \dfrac{1}{x}\) est continue sur l’intervalle \([k, k+1]\). D’autre part la fonction \(g: x\mapsto \dfrac{1}{k}\) est constante sur cet intervalle.

\sphinxAtStartPar
De plus, nous avons pour tout \(x\in [k, k+1], \dfrac{1}{x}\leq \dfrac{1}{k}\) c’est\sphinxhyphen{}a\sphinxhyphen{}dire, \(f\leq g\) sur cet intervalle.

\sphinxAtStartPar
Par suite, \(\int_k^{k+1}f(x)dx= \int_k^{k+1}\dfrac{dx}{x} \leq \int_k^{k+1} g(x)dx = \dfrac{1}{k}\).

\sphinxAtStartPar
Maintenant soit \(n\geq 2\). Alors
\begin{equation*}
\begin{split}
\int_1^{2}\dfrac{dx}{x}+ \ldots + \int_{n-1}^{n}\dfrac{dx}{x}=\int_1^{n}\dfrac{dx}{x}\leq 1+ \dfrac{1}{2} + \ldots + \dfrac{1}{n} =S_n
\end{split}
\end{equation*}
\sphinxAtStartPar
Nous avons \(\int_1^{n}\dfrac{dx}{x}=ln(n)\). Or, \(ln(n) \to +\infty\) donc \(S_n \to +\infty\).

\sphinxAtStartPar
Enfin, la série \(\sum \dfrac{1}{n}\) est divergente.


\subsection{Série géométrique \protect\(\sum_{n\geq 0} q^n\protect\)}
\label{\detokenize{series:serie-geometrique-sum-n-geq-0-q-n}}
\sphinxAtStartPar
La suite des sommes partielles est donnée par \(S_n = 1+q+q^2 + \ldots + q^n\).

\sphinxAtStartPar
Donc
\begin{equation*}
\begin{split}
S_{n+1}= 1+q+q^2 + \ldots + q^n + q^{n+1}=1+q(1+q+q^2 + \ldots + q^n)=1+qS_n
\end{split}
\end{equation*}
\sphinxAtStartPar
D’autre part,
\begin{equation*}
\begin{split}
S_{n+1}- S_n = q^{n+1}
\end{split}
\end{equation*}
\sphinxAtStartPar
Donc
\begin{equation*}
\begin{split}
1+(q-1)S_n=q^{n+1}
\end{split}
\end{equation*}
\sphinxAtStartPar
Si \(|q|\neq 1\),
\begin{equation*}
\begin{split}
S_n = \dfrac{q^{n+1}-1}{q-1}=\dfrac{1-q^{n+1}}{1-q}
\end{split}
\end{equation*}
\sphinxAtStartPar
et s’en suit alors que \(S_n\) diverge si \(|q| > 1\) et converge si \(|q| < 1\). Dans ce dernier cas on a \(Sn \to \dfrac{1}{1-q}\).

\sphinxAtStartPar
Enfin, la série diverge si
\(q = ±1\). En effet pour les deux cas, le terme général ne converge pa vers 0.

\begin{sphinxadmonition}{note}{Théorème}

\sphinxAtStartPar
La série \(\sum_{n\geq 0}q^n\) converge si, et seulement si, \(|q|<1\). Le cas échéant \(\sum_{n= 0}^{+\infty}q^n=\dfrac{1}{1-q}\)
\end{sphinxadmonition}

\begin{sphinxadmonition}{note}{Définition}

\sphinxAtStartPar
Soit \(\sum u_n\) une série convergente. On appelle reste d’ordre n de cette série
la quantité \(R_n\) donnée par
\begin{equation*}
\begin{split}
R_n = \sum_{k=n+1}^{\infty}u_k
\end{split}
\end{equation*}
\sphinxAtStartPar
On a alors
\begin{equation*}
\begin{split}
S= \sum_{i=0}^{\infty}u_k = S_n + R_n
\end{split}
\end{equation*}\end{sphinxadmonition}

\begin{sphinxadmonition}{note}{Proposition}

\sphinxAtStartPar
Le reste d’ordre n d’une série convergente \(\sum u_n \) tend vers 0, lorsque \(n\to +\infty\).
\end{sphinxadmonition}

\begin{sphinxadmonition}{note}{Proposition}

\sphinxAtStartPar
Soient \(\sum u_n\) et \(\sum v_n\) deux séries convergentes. Alors \(\sum (u_n + \lambda v_n)\) converge pour tout \(\lambda \in \mathbb R\).
\end{sphinxadmonition}

\begin{sphinxadmonition}{note}{Proposition}

\sphinxAtStartPar
Soient \(\sum u_n\) et \(\sum v_n\) deux séries convergentes telles que \(u_n \leq v_n\). Alors
\begin{equation*}
\begin{split}
\sum_{i=0}^\infty u_n \leq \sum_{i=0}^\infty v_n
\end{split}
\end{equation*}\end{sphinxadmonition}


\subsection{Séries à termes positifs}
\label{\detokenize{series:series-a-termes-positifs}}
\begin{sphinxadmonition}{note}{Lemme}

\sphinxAtStartPar
Soit \(\sum u_n\) une série a termes positifs, \(u_n\geq 0\). Alors \(\sum u_n\) converge si et seulement si la suite des somme partielles est majorée, c’est\sphinxhyphen{}a\sphinxhyphen{}dire \(\exists M>0\) tel que \(S_n=\sum_{k=0}^n u_k \leq M, \forall n\).
\end{sphinxadmonition}

\begin{sphinxadmonition}{note}{Théorème (Critères de convergence)}

\sphinxAtStartPar
Soient les deux séries \(\sum u_n\) et \(\sum v_n\). Alors on :
\begin{enumerate}
\sphinxsetlistlabels{\arabic}{enumi}{enumii}{}{.}%
\item {} 
\sphinxAtStartPar
Si \(0\leq u_n \leq v_n\), alors on a \(\sum v_n\) converge \(\Rightarrow \sum u_n\) converge et \(\sum u_n\) diverge \(\Rightarrow \sum v_n\) diverge.

\item {} 
\sphinxAtStartPar
si \(\dfrac{u_n}{v_n} \to 1\) lorsque \(n \to +\infty\) alors \(\sum u_n\) et \(\sum v_n\) sont de même nature.

\item {} 
\sphinxAtStartPar
S’il existe \(\alpha >1\) tel que \(n^\alpha u_n \to 0\) alors \(\sum u_n\) converge.

\item {} 
\sphinxAtStartPar
Soit \(u_n \geq 0\) telle que \(\sqrt[n]{u_n} \to l\) alors, \(\sum u_n\) converge si \(l<1\) et diverge si \(l>1\). (Règle de Cauchy).

\item {} 
\sphinxAtStartPar
Soit \(u_n>0\) telle que \(\dfrac{u_{n+1}}{u_n} \to l\), alors \(\sum u_n\) converge si \(l<1\) et diverge si \(l>1\). (Règle de D’Alembert).

\end{enumerate}

\sphinxAtStartPar
Pour le cas \(l=1\) dans les regles de Cauchy et D’Alembert on ne peut pas identifier la nature de la série. Dans ce cas on doit utiliser d’autres méthodes (déjà citées) ou d’étudier la suite des sommes partielles.
\end{sphinxadmonition}


\subsubsection{Exemple: Serie de Riemann}
\label{\detokenize{series:exemple-serie-de-riemann}}
\sphinxAtStartPar
On considere la serie \(\dfrac{1}{n^\alpha}\) avec \(\alpha \in \mathbb R\).
\begin{itemize}
\item {} 
\sphinxAtStartPar
Si \(\alpha \leq 0\) alors \(\dfrac{1}{n^\alpha}= n^{-\alpha} \nrightarrow  0\) donc la série \(\sum \dfrac{1}{n^\alpha}\) diverge.

\item {} 
\sphinxAtStartPar
Si \(\alpha =1\) alors il s’agit de la série harmonique qui divergente.

\item {} 
\sphinxAtStartPar
si \(0 < \alpha\),  on a \(\dfrac{1}{n}<\dfrac{1}{n^\alpha}\) donc la série \(\sum \dfrac{1}{n^\alpha}\) diverge d’après le critère de comparaison.

\item {} 
\sphinxAtStartPar
si \(\alpha > 1\), alors on a pour \(k\geq 2\):

\end{itemize}
\begin{equation*}
\begin{split}
\dfrac{1}{k^\alpha} \leq \int_{k-1}^{k} \dfrac{dx}{x^\alpha}
\end{split}
\end{equation*}
\sphinxAtStartPar
Donc
\begin{equation*}
\begin{split}
S_n= \sum_{k=1}^{n} \dfrac{1}{k^\alpha} \leq \int_1^n \dfrac{dx}{x^\alpha}
\end{split}
\end{equation*}
\sphinxAtStartPar
Or \(\alpha > 1\) donc \(\int_1^\infty \dfrac{dx}{x^\alpha}\) existe est \(\lim_{n \to +\infty} \int_1^n \dfrac{dx}{x^\alpha}= \int_1^\infty \dfrac{dx}{x^\alpha}\)

\sphinxAtStartPar
Alors, \(S_n\) est croissante et majorée donc convergente :
\begin{equation*}
\begin{split}
\sum_{n=1}^{+\infty} \dfrac{1}{k^\alpha} = \lim_{n \to +\infty} S_n \leq \int_1^\infty \dfrac{dx}{x^\alpha}.
\end{split}
\end{equation*}
\begin{sphinxadmonition}{note}{Théorème (Séries de Riemann)}

\sphinxAtStartPar
La serie \(\sum \dfrac{1}{n^\alpha}\), \(\alpha \in \mathbb R\), converge si et seulement si \(\alpha >1\).
\end{sphinxadmonition}


\subsection{Exemples}
\label{\detokenize{series:exemples}}\begin{enumerate}
\sphinxsetlistlabels{\arabic}{enumi}{enumii}{}{.}%
\item {} 
\sphinxAtStartPar
La série de terme général \(u_n =\dfrac{2^{2n}e^{-2n}}{n}\) (\(\sum \dfrac{2^{2n}e^{-2n}}{n}\))

\sphinxAtStartPar
Par la règle de D’Alembert, on a
\begin{equation*}
\begin{split}
    \dfrac{u_{n+1}}{u_n}=\dfrac{\dfrac{2^{2n+2}e^{-2n-2}}{n+1}}{\dfrac{2^{2n}e^{-2n}}{n}}=\left(\dfrac{2}{e}\right)^2 \dfrac{n}{n+1} \to \left(\dfrac{2}{e}\right)^2 <1
    \end{split}
\end{equation*}
\sphinxAtStartPar
Donc la série est convergente.

\item {} 
\sphinxAtStartPar
La série de terme général \(u_n = (\dfrac{n}{n+1})^{n^2}\)

\sphinxAtStartPar
Par la règle de Cauchy, on a
\begin{equation*}
\begin{split}
    \sqrt[n]{u_n} = \sqrt[n]{\left(\dfrac{n}{n+1}\right)^{n^2}} = \sqrt[n]{\left(\dfrac{1}{\dfrac{n+1}{n}}\right)^{n^2}} = \sqrt[n]{\left(1+\dfrac{1}{n}\right)^{-n^2}}=\left(\dfrac{1}{1+n}\right)^{-n}=\dfrac{1}{\left(1+\dfrac{1}{n}\right)^n}
    \end{split}
\end{equation*}
\sphinxAtStartPar
Or, \((1+\dfrac{1}{n})^n \to e\). Donc, \(\sqrt[n]{u_n}= \dfrac{1}{(1+\dfrac{1}{n})^n} \to \dfrac{1}{e} < 1\)

\sphinxAtStartPar
Par suite la série est convergente.

\item {} 
\sphinxAtStartPar
Soit \(u_n = (\dfrac{an}{n+1})^{n^2}\) avec \(a\in \mathbb R^+\). On veut étudier la nature de la série de terme général \(u_n\).

\sphinxAtStartPar
Par la régle de Cauchy, on a \(\sqrt[n]{u_n} = \sqrt[n]{\left(\dfrac{a}{1+\dfrac{1}{n}}\right)^{n^2}} = \left(\dfrac{a}{1+\dfrac{1}{n}}\right)^{n}= \dfrac{a^n}{1+\dfrac{1}{n}}\).
\begin{itemize}
\item {} 
\sphinxAtStartPar
si \(a<1\) nous avons \(\sqrt[n]{u_n} \to 0 < 1\), donc la série de terme général \(u_n\) converge.

\item {} 
\sphinxAtStartPar
si \(a=1\) nous somme dans le cas de l’exemple précédant. La série de terme général \(u_n\) converge.

\item {} 
\sphinxAtStartPar
si \(a>1\) nous avons \(\sqrt[n]{u_n} \to +\infty\), donc la série de terme général \(u_n\) diverge.

\end{itemize}

\sphinxAtStartPar
Par suite, la série \(\sum (\dfrac{an}{n+1})^{n^2}\) avec \(a\in \mathbb R^+\) converge si et seulement si \(a \leq 1\).

\item {} 
\sphinxAtStartPar
La serie de terme general \(u_n = \int_0^{\frac{\pi}{n}} \dfrac{\sin x}{1+x^2}dx\)

\sphinxAtStartPar
Pour tout \(x \in [0, \frac{\pi}{n}] \subset [0, \pi]\) on a \(1 \leq 1+x^2 \leq 1+ \pi^2\). Donc \(\dfrac{1}{1+\pi^2} \leq \dfrac{1}{1+x^2} \leq 1\). D’où,
\begin{equation*}
\begin{split}
    \dfrac{\sin x}{1+x^2} \leq \sin x
    \end{split}
\end{equation*}
\sphinxAtStartPar
et ca par ce que \(\forall x \in [0, \pi], \sin(x)\leq 0\). On en déduit,

\sphinxAtStartPar
\(0 \leq u_n =  \int_0^{\frac{\pi}{n}} \dfrac{\sin x}{1+x^2}dx \leq  \int_0^{\frac{\pi}{n}} \sin x dx = 1- \cos \frac{\pi}{n}\)

\sphinxAtStartPar
Et puisque \(\lim_{n\to +\infty} \dfrac{1- \cos \frac{\pi}{n}}{\frac{\pi}{n}} = 1\) (pourquoi?) donc la serie de terme general \(1- \cos \frac{\pi}{n}\) est de meme nature que la serie de terme general \(\dfrac{\pi^2}{n^2}\). mais cette serie est convergente car c’est une serie de Reimann (\(2>1\)). Donc \(1- \cos \frac{\pi}{n}\) est convergente et par compariason la serie de terme general \(u_n\) est convergente.

\end{enumerate}


\subsection{Séries alternées}
\label{\detokenize{series:series-alternees}}
\begin{sphinxadmonition}{note}{Théorème (Critère spécial des séries alternées)}

\sphinxAtStartPar
Soit \(\sum u_n\) une série alternée. Si la suite \((|u_n|)_n\) est décroissante et \(|u_n| \to 0\), alors \(\sum u_n\) converge.
\end{sphinxadmonition}

\begin{sphinxadmonition}{note}{Exemple}

\sphinxAtStartPar
La série de terme général \(u_n = \dfrac{(-1)^n}{n}\) est une série alternée. En effet, elle s’écrit comme \((-1)^n|u_n|\). La série n’est pas absolument convergente car \(|u_n|= \dfrac{1}{n}\). Donc on ne peut rien dire à ce stade. En revanche, on remarque que la suite \((|u_n|)_n\) est décroissante et \(|u_n| \to 0\). Donc, en appliquant le critère spécial des séries alternées la série de terme général \(u_n = \dfrac{(-1)^n}{n}\) est convergente.
\end{sphinxadmonition}


\section{Exercices}
\label{\detokenize{exo4:exercices}}\label{\detokenize{exo4::doc}}

\subsection{Exercice 1}
\label{\detokenize{exo4:exercice-1}}
\sphinxAtStartPar
Etudier la convergence des séries suivantes :
\begin{equation*}
\begin{split}
\sum_{k=1}^{+\infty} \dfrac{1}{2k} = \dfrac{1}{2} + \dfrac{1}{4}+ \dfrac{1}{6}+\ldots 
\end{split}
\end{equation*}\begin{equation*}
\begin{split}
\sum_{k=1}^{+\infty} \dfrac{1}{2k+1} = \dfrac{1}{3} + \dfrac{1}{5}+ \dfrac{1}{7}+\ldots 
\end{split}
\end{equation*}

\subsection{Exercice 2}
\label{\detokenize{exo4:exercice-2}}
\sphinxAtStartPar
Etudier la convergence des séries suivantes :
\begin{equation*}
\begin{split}
S_1 = \sum_{n=2}^{+\infty} \dfrac{n^2+1}{n^2}
\end{split}
\end{equation*}\begin{equation*}
\begin{split}
S_2 = \sum_{n=2}^{+\infty} \dfrac{2}{\sqrt{n}}
\end{split}
\end{equation*}\begin{equation*}
\begin{split}
S_3 = \sum_{n=2}^{+\infty} \dfrac{(2n+1)^4}{(7n^2+1)^3}
\end{split}
\end{equation*}

\subsection{Exercice 3}
\label{\detokenize{exo4:exercice-3}}
\sphinxAtStartPar
Etudier la convergence des séries suivantes :
\begin{equation*}
\begin{split}
S_4 = \sum_{n=2}^{+\infty} (1-\dfrac{1}{n})^n
\end{split}
\end{equation*}\begin{equation*}
\begin{split}
S_5 = \sum_{n=2}^{+\infty} (ne^\frac{1}{n}-n)
\end{split}
\end{equation*}\begin{equation*}
\begin{split}
S_6 = \sum_{n=2}^{+\infty} \ln (1+e^{-n})
\end{split}
\end{equation*}

\subsection{Exercice 4}
\label{\detokenize{exo4:exercice-4}}
\sphinxAtStartPar
Déterminer la nature des séries dont les termes généraux sont les suivants :
\begin{enumerate}
\sphinxsetlistlabels{\arabic}{enumi}{enumii}{}{.}%
\item {} 
\sphinxAtStartPar
\(u_n = \left(\dfrac{n}{n+1}\right)^{n^2}\)

\item {} 
\sphinxAtStartPar
\(u_n = \dfrac{1}{n\cos ^2 n}\)

\item {} 
\sphinxAtStartPar
\(u_n = \dfrac{1}{(\ln(n))^n}\)

\end{enumerate}


\subsection{Exercice 5}
\label{\detokenize{exo4:exercice-5}}
\sphinxAtStartPar
Justifier la convergence et calculer la somme des séries suivantes :
\begin{equation*}
\begin{split}
\sum_{n\geq 10}(\sqrt{2})^{-k}
\end{split}
\end{equation*}\begin{equation*}
\begin{split}
\sum_{k\geq 1}(\dfrac{e}{\pi})^k
\end{split}
\end{equation*}

\subsection{Exercice 6}
\label{\detokenize{exo4:exercice-6}}
\sphinxAtStartPar
On considère la suite \((u_n)_{n\in \mathbb N}\) définie par \(u_0=1\) et, pour tout \(n\in \mathbb N\), \(u_{n+1} =\dfrac{2n+2}{2n+5}u_n\).
\begin{enumerate}
\sphinxsetlistlabels{\arabic}{enumi}{enumii}{}{.}%
\item {} 
\sphinxAtStartPar
Montrer que la suite \((u_n)_n\) est convergente. On note \(l\) sa limite.

\item {} 
\sphinxAtStartPar
Montrer que la série de terme général \((\ln(u_n)- \ln(u_{n+1}))\) est une série divergente.

\item {} 
\sphinxAtStartPar
En déduire la valeur de \(l\).

\end{enumerate}


\subsection{Exercice 6}
\label{\detokenize{exo4:id1}}
\sphinxAtStartPar
Etudier la convergence des séries \(\sum u_n\) suivantes :
\begin{equation*}
\begin{split}
u_n = \dfrac{\sin (n^2)}{n^2}
\end{split}
\end{equation*}\begin{equation*}
\begin{split}
u_n =(-1)^n \dfrac{\ln (n)}{n}
\end{split}
\end{equation*}\begin{equation*}
\begin{split}
u_n =\dfrac{\cos (n^2\pi)}{n\ln (n)}
\end{split}
\end{equation*}






\renewcommand{\indexname}{Index}
\printindex
\end{document}